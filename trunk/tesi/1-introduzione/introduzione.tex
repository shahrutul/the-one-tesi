% this file is called up by thesis.tex
% content in this file will be fed into the main document

%: ----------------------- name of chapter  -------------------------
\chapter{Introduction}\label{introduzione} % top level followed by section, subsection


%: ----------------------- paths to graphics ------------------------

% change according to folder and file names
%\graphicspath{{2-Consorzi/images/}}


%: ----------------------- contents from here ------------------------
Since the origins of civilization, mankind has had the need to share informations between places very distant from each other. Communication means have evolved enormously through the centuries from the first letters carried by runners. Riding couriers moved mails through greater distances allowing at the same time a faster communication between far cities. The methods of communication kept changing and in the 1840's the electric telegraph was invented and messages were sent much quicker through the first invisible means of communication. Just twenty years later, the first transoceanic submarine cable had been laid, and it has been possible to instantaneously send messages between America and Europe. Next, the telephone was invented and people began to talk from great distances as if they were sitting in the same room. Finally, with the rise of the Internet a new world-scale communication became possible.
\\

In the last years the access to the internet has become available for a large number of people, using many different types of devices to connect to the network. Fifteen years ago internet access was a prerogative of wired connected computers and cellular phones were devices only able to make phone calls and send/receive text messages. Today, smartphones are becoming very popular and these devices evolved from simple mobile phones to devices able to browse into the internet, read emails and watch video streaming from the net.
\\

All this new possibilities issued new problems regarding the related request for connectivity. While the network's core is highly connected and well suited for routing via
conventional routing algorithms, mobile devices often work in environments with infrastructures that suffer from intermittent connectivity and difficult to predict changes in topology. In these contexts, traditional routing algorithm designed for wired, high connected networks, lack in efficiency and other strategies must be adopted.
\\

One cause of intermittent connectivity is user movement. While mobile device's users moves, the network topology changes in ways that are difficult to predict, connection links are broken and new connections are established. In this context could be useful to use overlay networks like Delay Tolerant Networks (DTNs). These networks were initially designed for deep-space communication and one of the main characteristics is to adopt an asynchronous communication method named store-and-forward. This method, borrowed from Old West pony expresses, takes advantage of nodes mobility making them carry data along their path, increasing the communication range of the single node.

%se voglio aggiungere prima descrizione di m2mshare


\section{Contributions}
In this thesis, we evaluate the performance of M2MShare, a protocol which implements DTN techniques in mobile phones world, in order to enable a peer-to-peer file sharing system between peers in the network using their mobility and opportunistic contacts among them. To do so we implement the protocol in the Oppostunistic Network Environment (ONE) simulator. This simulator is able to emulate human movement adopting several movement models in a map-based environment. To our judgement, our contributions are the following:
\begin{itemize}
\item implementing M2MShare and evaluating its behaviour using a simulator (the ONE simulator) able to emulate nodes movement in a realistic urban scenario.
\item evaluating the efficiency of the new paradigm created by M2MShare of use P2P solutions that matches file sharing with mobile users, allowing them to exchange files with each other.
\item evaluating the efficiency of using task delegations to dynamically establish forward routes along the destination path in the network.
\item evaluating the efficiency of the new file division strategy introduced with M2MShare 
\item enhancing M2MShare adding support to multi-hop delegations in order to further increase the search area for a single node to other disconnected overlay networks.
\item enhancing the ONE simulator adding some new features to it and then making them available to the simulator users community\footnote{The ONE simulator user contributions \href{http://www.netlab.tkk.fi/tutkimus/dtn/theone/\#contrib}{http://www.netlab.tkk.fi/tutkimus/dtn/theone/\#contrib}}.
\end{itemize}

\section{Document outline}
The remainder of this document is organized with the following chapters:
\begin{enumerate}
\setcounter{enumi}{1}
\item \textbf{Background:} in this chapter we describe some must-know technologies, useful to the reader before to read the following chapters. We describe Delay Tolerant Networks (DTN) and Peer-to-Peer (P2P) systems.
\item \textbf{M2MShare:} this chapter describes M2MShare, the protocol object of the study and simulations of this thesis.
\item \textbf{Movement models:} in this chapter we give an overview about movement models used to simulate people movements. We describe the characteristics of several movement models, comparing them, and finally we describe the model we use for our simulations.
\item \textbf{The ONE simulator:} this chapter presents the ONE (Opportunistic Network Environment), the simulator we used for our evaluation of M2MShare.
\item \textbf{Implementation:} in this chapter we describe our implementation of M2MShare into the ONE and the improvements we did to the simulator to completely emulate the protocol behaviour.
\item \textbf{Simulations and results:} in this chapter we present the simulations we did. For each simulation we describe the settings, the protocol aspects evaluated, and finally we give the related results with a graphical representation.
\item \textbf{Conclusions:}  this chapter gives the concluding remarks, and proposes some related future works and enhancements. It also contains the due acknowledgements.

\end{enumerate}
 
% ---------------------------------------------------------------------------
%: ----------------------- end of thesis sub-document ------------------------
% ---------------------------------------------------------------------------

