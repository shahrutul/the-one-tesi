% this file is called up by thesis.tex
% content in this file will be fed into the main document

%: ----------------------- name of chapter  -------------------------
\chapter{Movement models}\label{movimento} % top level followed by section, subsection


%: ----------------------- paths to graphics ------------------------

% change according to folder and file names
%\graphicspath{{2-Consorzi/images/}}


%: ----------------------- contents from here ------------------------
in dtn no connessioni end to end
mobilità dei nodi influenza protocolli di routing
importanza di simulare il protocollo per valutare la rete globalmente
modelli di mobilità usati per simulare movimento dei singoli users 
influenzano posizione e velocità
desiderabile che simulino il comportamento dei nodi il più realisticamente possibile
senza soffermarsi a simulare un singolo caso specifico



One important aspect to consider to achieve a good level of realism in DTN simulations is to simulate movement of nodes inside the simulated world. In Delay Tolerant Networks, movement of nodes is essential for the performance of the network, since end-to-end connectivity does not always exist and packets are delivered in a store and forward manner. For this reason is fundamental to simulate, with a good level of realism, movements behaviour of nodes inside the simulation.
\\

Capturing movement accurately in the real usage scenarios is also needed for a reliable assessment of a new protocol. Movement can be captured in simulations using real movement traces or synthetic traces generated by a movement model.

\section{Random Walk}
The Random Walk movement model (RW) is one of the simplest mobility models available. In this model the simulation area is an open empty space where nodes move. Each node choose randomly an angle and a speed and for the next interval it moves using that speed and direction values. The choice is done for every simulation instant. The Random Walk model is a memoryless mobility model, because information about the previous status is not used for the future decision. 


\section{Random Waypoint}
The Random Waypoint movement model (RWP) is a generalization of RW and is one of the most used in mobility simulations. In this model every node choose a random location inside the simulation area and moves toward it at a random speed uniformly chosen from $(V_{min}, V_{max}]$, where $V_{min}$ is the minimum and $V_{max}$ is the maximum speed of the simulation. When the node reaches its destination, it waits for a random amount of time and then repeats the previous steps. This model has the shortcoming, shown in \cite{RWPHarmful}, that the simulation do not reach a steady state and average speed of nodes constantly decreases if $V_{min}$ value is set to zero, as in default implementation of the model. 


\section{Random Map-Based Movement}
The simplest map-based mobility model is Random Map-Based Movement (MBM). Nodes adopting this model moves randomly but always in streets described in the map. This is done walking from one map node, i.e. an address on a street or a crossroad, to the other by always randomly selecting one of the directly connected map nodes. Result of this behaviour is a random movement in which there are more contacts between nodes than in Random Walk model. This is due to the restriction to node movement to follow streets described in the map.


\section{Shortest Path Map-Based Movement}
Shortest Path Map-Based Movement (SPMBM) is another mobility model that uses a map-described environment to restrict node movement. With this model, nodes chooses their destination randomly inside the map, then calculate the shortest path to reach it and finally walk along this path. Dijkstra's algorithm is used to calculate the path. Destinations can be chosen in a totally random way or from a set of Points of Interests (POI). This can be useful to emulate interesting places like restaurants, monuments or shops. This model adds one level of realism, compared to MBM, since nodes don't move in a completely random way, but they have destinations and follow the best path to reach it.


\section{Routed Map-Based Movement}
A movement model in which nodes movements are completely determined is Routed Map-Based Movement (RMBM). In this model nodes moves following predetermined routes, for the duration of the simulation. Nodes adopting this model can easily emulate the scheduled and repetitive movement of public transport, e.g. buses, trams or trains.


\section{Working Day Movement Model}
\label{descrWDM}
Movement models previously described are simple to understand and implement, but they don't completely capture such characteristics like heterogeneous behaviour, relationships between users or repetitive behaviour. Working Day Movement Model (WDM) \cite{articoloWdm} does this using a simple idea: people follow certain routines during a typical working day, i.e. go to work, go out with friends, etc. 
\\

All these actions make a person to meet usually the same group of people. This reflect heterogeneity in user movement, i.e. one node will hardly meet all other nodes in the simulation. This also raise locality in nodes movement, reducing encounters between nodes.
\\

During a typical working day emulated by this model, a node is involved in three main activities:
\begin{itemize}
\item sleep at home
\item work in the office
\item go out in the evening with friends
\end{itemize}
These activities can obviously vary from person to person, depending on the lifestile of the subject. These are the most generic and can be associated to the typical working day of a large amount of people.
\\

A person usually does these activities in the same locations, i.e. it sleep in its home and works in the same office. It can be assumed that when it goes out with friends, they will go to a set of favourite places. These activities are repetitive, day by day, and are performed in places in common with other people. Simulating this, allow the spontaneous creation of users communities, e.g. people living in the same home location are a family, users working in the same office are colleagues and friends will meet to go together to their favourite places in the evening. A movement model able to emulate all these behaviours is fundamental to evaluate a protocol like M2MShare, where the frequency of encounters between nodes influences the routing strategy.\\

Creation of such communities of users, and in general relationships between nodes, are not shown using simpler movement models like RMBM or SPMBM i.e. in a longer period of time, each node is equally likely to meet any other node an equal number of times. In the Random Walk movement model, nearby nodes are more likely to get in contact and there are no clear relationships between nodes in most of the simple models.
\\

To emulate the three main activities, WDM uses simpler mobility models as submodels. A node moving using WDM, will change several submodels during a single simulated day. Submodels are also used to simulate different modes used by nodes to move through the map. A person can so move from a location to another walking, driving his car (if it has one) or using public transport.
\\
%Per la simulazione delle attività giornaliere, WDM utilizza dei sottomodelli dedicati, oltre a dei sottomodelli preposti a simulare gli spostamenti fra un'attività e l'altra. Una persona si potrà quindi spostare a piedi, in auto o utilizzando i mezzi pubblici, a seconda della propria disponibilità e convenienza. Il fatto di muoversi da soli o in gruppo (prendendo lo stesso bus o camminando assieme la sera) permette di avere dei comportamenti eterogenei e quindi migliorare ulteriormente la realisticità degli spostamenti compiuti dai vari nodi.


\subsection{Example of working day}
During one typical working day, the starting location of a node is its own house. Every node has one waking up time, indicating what time that node leaves home in the morning. The value is drawn from a normal distribution with mean 0 and standard deviation configurable. The wake up time value is generated at the beginning of the simulation and remains the same for all the simulation duration, i.e. every node leaves home at same time every day. Different values between nodes emulate differences in lifestyles, e.g. one person can be faster to get dressed in the morning and another can be slower.
\\ 

When a node leaves home, it travels with destination its office. This can happen walking, driving its own car (if available) or using public transport. The choice of transportation is done evaluating availability and speed to reach the destination. Accordingly to the chosen transportation is used the relative mobility submodel.
\\

Once the node reaches its office, it stays there for all the working time, which is a configurable value. After the working hours, the node decide, by drawing, whether it returns home or goes out for the evening activity. Different submodels are used again for transitions between the locations.


\subsection{Home Activity Submodel}
Every node has one location in the map set as Home Location. Home location is randomly chosen, for every node, at the beginning of the simulation between a set of geographical points in the related district of the map. The chosen location remains the same for all the simulation duration, i.e. every node wake up in the morning and returns in the evening to the same location.
\\

When the node returns home it moves near this location for a while and then it stays still until next wake up time. This behaviour is not an error, but it is done to emulate the activity to left the mobile phone in the same place inside the house while the user does typical domestic activities which do not involve the use of mobile phone, e.g cooking, watching TV or sleeping.

\subsection{Office Activity Submodel}
The submodel related to the working activities is a model which emulate the behaviour of an employee inside its office. During the working time the employee stays at his desk for most of the time but it can leave it for attend a meeting, talk with a colleague of have a break. During all these moments it can meet nodes related to other colleagues.
\\

The office is described as a single room with rectangular shape, where the only entrance point, the door, is placed in the upper left hand corner. Every employee has its own desk, located in the same place inside the office for all the simulation duration. In this model is not simulated the presence of walls or other furniture inside the office. To compensate for this, the office is described as a room bigger than the normal size, to emulate time needed to avoid obstacles and reach a destination when moving inside the office.
\\

When entered in the office, the employee moves directly to its desk and then stays there for an amount of time drawn from a Pareto distribution. When this period is passed, the node chooses a random destination inside the office, walks to reach it and then waits for an amount of time drawn from the same Pareto distribution, before to return to its desk. The repetition of this movements and pauses from the desk to some random locations inside the office continues until working time expires. 
\\

The working time length is a configurable parameter, as are configurable distribution parameters. This allow to tune movements inside the office as there are a variety of different jobs and buildings where people move according to different patterns e.g. a teacher which moves every hour to change class inside a school or a cashier which do not leave its place for all the working time.


\subsection{Evening Activity Submodel}
The Evening Activity Submodel emulates activities performed after work, in the late afternoon or in the evening, like shopping, go eat in a restaurant or in a pizzeria. These activities are performed in group and with a configurable probability which determines if a person returns directly home, after work, or stays out with friends.
\\

When working time is expired, the node moves to one of its favourite places using one transport submodel. When arrived, it waits for other people until there are enough people to create a group and start the activity. The minimum number of people needed to create a group is configurable and when all groups for a place are full and a new person arrives, a new group is created.
\\

When a group is full, all people in the group walk together to reach a near, randomly chosen, location and they wait for a random amount of time, drawn from a range of values. This waiting time emulates the duration of the dinner, shopping or a movie at cinema and when it is over all nodes belonging to the group leave each other and return to their own houses.

\subsection{Transport Activity Submodel}
Transport Activity Submodel used by nodes to move between home, office and evening activities.
\\

At the beginning of the simulation, to every node is assigned a car with configurable probability. Nodes with cars moves always using it, while people which do not have one walks or uses public transport. A model which uses different types of transport models adds additional heterogeneity and this influences also routing protocols, since quicker nodes, like ones using cars for instance, can quickly transfer packets for longer distances.
\\

According to the chosen transportation, Transport Activity Submodel uses three different submodels:
\\

\begin{description}
\item [Walking Submodel]: nodes which do not have a car walks on the streets toward the destination at constant speed. They uses Dijkstra algorithm to find the shortest path from the current position to the destination.

\item [Car Submodel]: nodes with a car move faster than walking nodes, driving between a main activity and the next one, but walks during a single activities, i.e. they do not use the car inside the office, of course. In this model are not considered traffic congestions, traffic lights and every car carries one single person.

\item [Bus Submodel]: there can be several routes of public transport (buses, trams, trains) in the simulated map. Each route is run by several public transport according to a schedule. Each public transport can carry more than one person at a time.
\end{description}

Every person which do not have a car knows one bus route and can use every bus running on that route. The choice between walking or using public transport is done considering a difference between Euclidean distances. If the distance between node's location and the nearest bus stop summed with the Euclidean distance from the destination to the nearest bus stop is shorter than the Euclidean distance between the node's location and the destination, than the node uses the public transport system, otherwise it walks to the destination. If public transport is chosen the node changes between two transport submodels: it walks to the nearest bus stop using Walking Submodel and, switching to Bus Submodel, it waits for the bus. When the bus arrives, the node enters it and travels until to the bus stop closer to the destination. Finally it switches back to the Walking Submodel to reach the destination.
 
 


\section{Real user traces VS Synthetic models}
The most realistic user movement or contact patterns are the ones that happen in the real world. Therefore, several studies have focused on tracking user movement to be able to use the traces later directly for simulations or to learn more about which characteristics are common in real user behaviour.
Real movement traces have usually been obtained by analysing WLAN access point data or
having users carry around devices equipped with GPS modules. Contact traces have
mostly been collected from real world experiments where users have been carrying
around Bluetooth devices tracking other devices within range.

There are some problems about directly using the collected traces in simulations. Real world traces are usually from very specific environments like university campuses areas. These traces are not so generic to be used to simulate movement in a different environment, like a centre of a city, because behaviour and movement patterns of interested people are different. \\

Real trace are generally been obtained by analysing WLAN access point data or
having users carry around devices equipped with GPS modules. In the first case only certain areas have WLAN coverage and so movements outside access points communication range are not recorded. Using GPS modules limits mobility recording only to outdoor movements. 


When the locations of users are derived from access point data, the result is roughly building level granularity [16]. This is because users are not always connected to the closest access point and the movement between access points is difficult to capture. Most of the devices connecting to WLAN access points in these experiments are laptops and PDAs which are not always carried by the users and are not necessarily always turned on. Whether this characteristic is wanted in the simulations depends on what is modeled and simulated. Paper \cite{ImpactofHumanMobility} argues that the on/off times are an important characteristic of wireless users that needs to be taken into account and modeled for simulations. 

It is also practical to have a model with configurable parameters to work with. Sometimes a protocol or an application developer wants to test how the protocol or application performs when certain parameters change, to better be able to find out weaknesses and strengths. Sometimes researchers also need a very simple model to work with, to be able to use it in mathematical proofs for various theorems.


Synthetic models are often preferred since they are easier to work with than real user traces. Moreover, real user traces are rarely available for the environment to be modelled. Additionally, researchers want to do sensitivity analysis to find out how protocols and applications perform under different conditions. This is not possible with real user traces unless a parameterized model has been successfully extracted. 
There are two types of synthetic movement models that have been proposed for these
analyses — generic high level models that aim to produce movement accurate enough
with statistical measures, and models that describe incidental scenarios, hoping for a
more accurate depiction of single devices.
While efficient to use in simulations, the high level models, such as Random Waypoint
[18], often imply that the scenarios for which the protocols are simulated have huge
numbers of nodes, so that the relevant protocol features are given statistically realistic
distributions of events. For scenarios with few nodes, the differences between different
usage scenarios become more significant. Thus, movement models that depict more
precisely some specific types of movement are needed.
Traditionally, the approach to create a movement model has been to identify a certain
characteristic of mobility and to create a mathematical model describing the movement
at a high level. Such characteristics can be speed distribution, social relationships
between nodes or favorite locations nodes will visit. These types of movement have
very few details and the movement is homogeneous in the sense that every node is
moving according to the same rule.

Before we go any further, we introduce two new terms: locality and heterogeneity. By
locality we mean the tendency of nodes spending most of their time within a small area.
Thus, a movement model where each node's movement is restricted to a small area has
high locality. By heterogeneity, we mean different movement patterns and properties
between nodes. Measuring heterogeneity in a specific context is not straightforward,
but in our experiments we will later on talk about differences in contact patterns in
terms of how often a node encounters new nodes compared to the fraction of earlier
encountered nodes.

Hsu and Helmy [26] show by studying real user traces that nodes are very often turned
on/off and only visit a small portion of the WLAN access points in campus areas.
Moreover, they find that node mobility while using network is very low and one node
only meets a small portion of all other nodes in the area. These types of characteristics
are usually not captured in movement models. Furthermore, they reveal repetitive
patterns with a period of one day and heterogeneity among nodes. Although
heterogeneity and repetitiveness has been modeled, most simple movement models do
not. According to Hsu and Helmy, the biggest issue with most synthetic models is that
they are not capturing such characteristics as heterogeneous behavior, switching
devices on/off or relationships between users.
We noticed that even though most of the known features of movement have been
modeled, no model exist where all or many of the features have been combined. Our
approach is to combine these different elements to create a new movement model.

% ---------------------------------------------------------------------------
%: ----------------------- end of thesis sub-document ------------------------
% ---------------------------------------------------------------------------

