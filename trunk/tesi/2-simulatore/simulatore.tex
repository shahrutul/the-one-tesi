% this file is called up by thesis.tex
% content in this file will be fed into the main document

%: ----------------------- name of chapter  -------------------------
\chapter{The ONE simulator}\label{simulatore} % top level followed by section, subsection


%: ----------------------- paths to graphics ------------------------

% change according to folder and file names
\graphicspath{{2-simulatore/img/}}


%: ----------------------- contents from here ------------------------
The simulation environment chosen for our simulations is The ONE (Opportunistic Network Environment) version 1.4.1\footnote{The ONE simulator \href{http://www.netlab.tkk.fi/tutkimus/dtn/theone/}{http://www.netlab.tkk.fi/tutkimus/dtn/theone/}}, described in \cite{articoloONE}. This simulation environment written in Java is completely configurable and is able to simulate completely the behaviour of nodes in the simulations, including movement, connections and to give a visual feedback about all these factors using its GUI. A more interesting feature, for our purposes, is the capability of emulate message routing using different routing protocols.
\\

To emulate nodes movement, the ONE can take in input traces from real-world measurements, from external mobility generators or it can generate movement patterns using synthetic mobility model generators. Differences between several available mobility are explained in Section \ref{movimento}. Either mobility models that routing protocols are managed like independent modules, which are dynamically loaded depending on what set in configuration files. This allow a relatively easy implementation of new mobility models and routing protocols in the simulator.
\\

The simulator finally allow to save data of interest from the completed simulations into report files. This reports are created using modules dynamically loaded in the same way to what happens with movement and routing modules.
\\

\section{Configuration}
\label{configurazioneONE}
A single simulation is set, before running, using configuration files which describe the simulated environment, from the simulation length to number of nodes emulating people in the simulated environment. Configuration files are text-based files that contain parameters about the simulation, user interface, event generation, and reporting modules. All these parameters are loaded before the beginning of simulations and are used to adjust details about the behaviour of loaded modules.
\\

Many of the simulation parameters are configurable sepa-
rately for each node group but groups can also share a set of param-
eters and only alter the parameters that are specific for the group.
The configuration system also allows defining of an array of val-
ues for each parameter hence enabling easy sensitivity analysis: in
batch runs, a different value is chosen for each run so that large
amounts of permutations are explored.

Inside configuration files, parameters are saved as key-value pairs and syntax for most of the
variables is:

\begin{center}
\textit{Namespace.key = value}
\end{center}

Namespace defines the part of the simulation environment where the setting has effect on. Many, but not all, namespaces are equal to the class name where they are read. This convention is especially followed by movement models, report modules and routing modules, so also created new modules should follow it.
\\

To make human-readability and configuration easier, numerical values can use suffixes kilo (k), mega (M) o giga (G), with \textquotedblleft .\textquotedblright as decimal separator. Boolean parameters accept \textquotedblleft true\textquotedblright
 or \textquotedblleft 1\textquotedblright , \textquotedblleft false\textquotedblright or \textquotedblleft 0 \textquotedblright .
\\
 
Comments can be inserted in setting files can contain comments too with \textquotedblleft
\#\textquotedblright character. Rest of the line is skipped when the settings are read.
\\

Every simulation can be set using several configuration files, to divide parameters into different categories, e.g. a file can contain scenario parameters, like maps, streets and districts, another file can contain parameters used to configure nodes, another file can contain reports configuration and so on. First configuration file read, if exists, is always \textquotedblleft default\_settings.txt\textquotedblright. Other configuration files given as parameter can define more settings or override some (or even all) settings in the previous files. The idea is that you can define in the earlier files all the settings that are common for
all the simulations and run different simulations changing parameters only in latests configuration files.
\\

The basic agents in the simulator are called nodes. A node models a mobile endpoint capable of acting as a store-carry-forward router (e.g., a pedestrian, car or tram with the required hardware). Nodes in the simulation world are divided into groups, each one configured with different capabilities. Inside a group, every node has the same characteristics, which are radio interface, persistent storage, movement, energy consumption and message routing protocol. It is possible to configure some of these capabilities as common for all groups, avoiding to set them individually for each group. Changing some configuration value between a group and another, allow to get heterogeneity between the behaviour of nodes in simulated scenario.
\\

In configuration files is also possible to provide array of settings for every parameter. This allows to run large amounts of different simulations using only single configuration file. Every simulation will be different from the previous and this is very useful to gather reports about different aspects in a common scenario.
\\ Syntax for these configuration parameters is

\begin{center}
\textit{Namespace.key = [run1_value; run2_value; run3_value; etc]}
\end{center}

Some parameters, finally, accept a file path as value and this can be either an absolute or relative path. These parameters are used for maps, nodes paths or events to be loaded by event generators modules.

\section{Visualization}

La principale modalità di visualizzazione fornita da ONE è quella tramite la GUI, che permette di seguire in tempo reale l'avanzamento della simulazione. Nella finestra principale è possibile osservare i movimenti dei vari nodi e, selezionandone uno specifico, ottenere informazioni riguardo le connessioni attive, i messaggi trasportati e altri dettagli. E' disponibile inoltre un riquadro in cui viene costantemente aggiornato un log di eventi generati durante la simulazione che possono essere filtrati a seconda di ciò che più interessa (ad esempio visualizzare solo le nuove connessioni o gli scambi di messaggi).
\\
\figuremacro{Schermata-ONE}{Schermata Principale}{La schermata principale del simulatore ONE}{}

Nel caso di modelli di movimento basati su una mappa, selezionando un nodo sarà possibile vedere il percorso seguito, la destinazione da raggiungere e ottenere informazioni avanzate riguardanti lo stato di quel nodo (connessioni, messaggi trasportati, ecc), come si può vedere in Figura \ref{Routing-Info}. La visualizzazione è personalizzabile zoomando, modificando la velocità di avanzamento e anche inserendo nello sfondo un'immagine, come ad esempio una carta stradale o una fotografia satellitare della zona interessata.
\\
\figuremacro{Routing-Info}{Routing Info}{Un esempio di finestra in cui sono presenti i dettagli relativi allo stato di un nodo}{}


L'altra modalità di seguire l'avanzamento di una simulazione è la lettura dei reports generati dai vari moduli durante l'esecuzione. Come i modelli di movimento e i protocolli di routing, i generatori di reports sono caricati dinamicamente all'avvio della simulazione, a seconda di ciò che è stato impostato nei files di configurazione.
Questa modalità di visualizzazione è particolarmente utile quando non si utilizza la GUI, ma si eseguono più simulazioni in batch, ottenendo quindi alla fine i report con i dati raccolti durante le varie simulazioni.
\\

La modalità batch è indicata nel caso si debbano eseguire più simulazioni in serie senza essere interessati a seguirne l'avanzamento in modalità grafica ma piuttosto valutandone i risultati una volta completate. In questo caso si dimostra molto utile la possibilità di specificare array di valori per i parametri di configurazione, in modo da poter programmare in anticipo le differenze fra le varie simulazioni della serie. Una volta avviato quindi il batch di simulazioni, ONE si occuperà si eseguirle in sequenza e alla massima velocità permessa della macchina in uso e salverà i reports generati in più files impostati durante la configurazione.

\section{Reports}
Il simulatore può gestire la generazione di più reports relativi alla simulazione in esecuzione. Questi report vengono creati da dei moduli attivati in fase di configurazione e consistono generalmente in files di testo in cui vengono salvati i dati e statistiche che verranno poi analizzati a simulazione terminata. Nella versione 1.4.1 di ONE, il sistema permette di generare reports relativi a:
\begin{description}
\item[messaggi,] che includono numero di messaggi creati, scambiati, scaduti, ecc
\item[contatti,] in cui viene indicato il contact e l'inter-contact time fra i vari nodi, oltre al totale dei contatti durante la simulazione.
\item[connessioni,] che descrivono l'alternarsi di stato delle connessioni 
\end{description}

Come per le altre parti di cui ONE è composto, anche i generatori di reports vengono gestiti come moduli, ed è quindi possibile aggiungerne di nuovi a seconda delle esigenze e dei dati da raccogliere nella singola simulazione.


\section{Esecuzione}
\label{esecuzioneONE}
Vale la pena di soffermarsi sull'esecuzione di una singola simulazione, vedendo quindi quali sono i vari passi che vengono eseguiti.
\\

La prima azione svolta dal simulatore è quella di caricare le impostazioni dai vari files di configurazione, la cui posizione viene passata per parametro al momento dell'esecuzione del simulatore. Man mano che un nuovo file di configurazione viene letto, i valori dei parametri in esso contenuto vanno ad impostare il valore di alcune variabili dell'ambiente di simulazione, sovrascrivendone anche il valore nel caso fossero già state impostate.
\\

Una volta caricate le impostazioni relative alla simulazione, viene creato lo Scenario. Questo contiene al suo interno tutti gli elementi attivi durante la simulazione (come i nodi, i generatori di reports e quelli di messaggi), così come quelli passivi (ad esempio le mappe che compongono il mondo simulato). In questa fase vengono quindi creati tutti i nodi partecipanti alla simulazione, ognuno dotato di un proprio modello di movimento, un router configurato secondo le caratteristiche del gruppo di nodi e una serie di \textit{listener} per la cattura di eventi e la successiva generazione di reports.
\\

Quando tutte le entità necessarie all'esecuzione sono stati create e configurate, si passa all'esecuzione vera e propria.
Questa consiste nel ripetere l'aggiornamento dello stato del mondo, chiamando un metodo \textit{update()}, e incrementare il valore del tempo simulato fino al raggiungimento di un tempo impostato come fine simulazione. L'incremento temporale che viene effettuato ad ogni aggiornamento è impostato nei files di configurazione (con il parametro \textit{Scenario.updateInterval}), espresso in secondi, ed influenza i vari modelli di movimento dei nodi. La prima operazione svolta durante l'\textit{update()} del mondo simulato è lo spostamento dei vari nodi, che avviene a seconda del modello di movimento adottato dal nodo e dell'incremento temporale applicato. Ad esempio un nodo che simula un automobilista si sposterà maggiormente rispetto ad uno che simula un pedone, a parità di intervallo di tempo simulato. Come vedremo nella sezione \ref{movimento} relativa ai modelli di movimento, questi possono essere anche molto complessi e simulare diversi comportamenti a seconda delle configurazioni adottate.
\\

Una volta effettuato il movimento, per ogni nodo viene aggiornato lo stato delle connessioni e del router simulato. Per ogni interfaccia di rete disponibile viene quindi aggiornato lo stato a seconda che lo spostamento abbia comportato una caduta della connessione o abbia permesso di entrare nel raggio di comunicazione di un interfaccia di rete relativa ad un altro nodo. Ogni qualvolta lo stato di una connessione cambia, vengono avvisati i \textit{listeners} interessati, per la generazione di report, e viene aggiornata la visualizzazione grafica della connessione, se attiva, come si può vedere, ad esempio, in Figura \ref{connessioni}.
\\
\figuremacro{connessioni}{connessioni}{Un esempio di connessioni tra nodi tratto dalla finestra principale di ONE. Nella parte sinistra dell'immagine si può notare la connessione attiva, mentre nella parte a destra i due nodi non sono più l'uno nel raggio di comunicazione dell'altro, indicato dal cerchio verde attorno al nodo.}{}

L'ultima parte dell'aggiornamento relativo allo stato di un nodo riguarda l'aggiornamento del router. Questo è fortemente dipendente dal protocollo di routing implementato ed è proprio nell'esecuzione del metodo \textit{update()} relativo al router che si svolgono le azioni caratteristiche di un protocollo rispetto ad un altro.

\section{Casualità}
come viene introdotta della casualità nelle varie simulazioni

\section{Limitazioni}
\label{limitazioniONE}
non più in basso del livello di routing
simulazione temporale discreta
mancanza di simulazione di un file system

\section{The map}
\label{mappaONE}
qua ci va la descrizione della mappa con la divisione in distretti e magari un'immagine delle rotte dei bus e del modello di movimento
% ---------------------------------------------------------------------------
%: ----------------------- end of thesis sub-document ------------------------
% ---------------------------------------------------------------------------

