\documentclass[11pt,a4paper]{book}		%grandezza carattere,formato a4 (stile)

\include{Latex/Macros/MacroFile1}

%\usepackage[italian]{babel}			%libreria per scrivere in italiano
\usepackage[english]{babel}				%libreria per scrivere in inglese

\usepackage[T1]{fontenc}
\usepackage[utf8]{inputenc}
\usepackage{listings}
%\usepackage[latin1]{inputenc}			%libreria per accettare i caratteri digitati da tastiera
\usepackage{fancyhdr}					%libreria per impostare il documento
\usepackage{indentfirst}				%indentazione all'inizio dei capitoli
\usepackage{graphicx}					%libreria per inserire grafici
\usepackage{newlfont}
\usepackage[hang,small,bf]{caption}		%font particolari ad esempio \textsc{}
\usepackage{amssymb}					%librerie matematiche
\usepackage{amsmath}					%librerie matematiche
\usepackage{latexsym}					%librerie matematiche
\usepackage{amsthm}						%librerie matematiche
                               			%libreria per la bibliografia
\usepackage{url}
\usepackage{subfigure}	
\usepackage{multirow}	

\oddsidemargin=30pt \evensidemargin=20pt	%impostano i margini

\hyphenation{sil-la-ba-zio-ne pa-ren-te-si DTN-Pen-ding-Down-lo-ad}        
%serve per la sillabazione: tra parentesi vanno inserite come nell'esempio le parole che latex non riesce a tagliare nel modo giusto andando a capo.

\pagestyle{fancy}\addtolength{\headwidth}{20pt}		%comandi per l'impostazione
\renewcommand{\chaptermark}[1]{\markboth{\thechapter.\ #1}{}}	%della pagina, vedi il manuale
\renewcommand{\sectionmark}[1]{\markright{\thesection \ #1}{}}	%della libreria fancyhdr
\rhead[\fancyplain{}{\bfseries\leftmark}]{\fancyplain{}{\bfseries\thepage}} %per ulteriori delucidazioni
\cfoot{}

\linespread{1.0}		%comando per impostare l'interlinea

\usepackage{hyperref}	%comando per i collegamenti e dettagli sui pdf		
\hypersetup{pdfauthor={Daniele Bonaldo},% autore del documento
pdftitle={Tesi di laurea magistrale},% titolo del documento
bookmarks=true,%mostra la barra dei collegamenti quando visualizza
colorlinks,% colora i links dei colori seguenti
citecolor=black,%
filecolor=black,%
linkcolor=black,%
urlcolor=black,%
pdftex}

%ambiente personalizzato per il codice sorgente
\newenvironment{mylisting}
{\begin{list}{}{\setlength{\leftmargin}{1em}}\item\scriptsize\bfseries}
{\end{list}}

\newenvironment{mytinylisting}
{\begin{list}{}{\setlength{\leftmargin}{1em}}\item\tiny\bfseries}
{\end{list}}
												

\begin{document} 	%inizio del documento

%%%%%COPERTINA%%%%%%%%%%%%%%%%%%%%%%%


%modificare questo file inserendo i propi dati
\begin{titlepage}
    \begin{center}
        {\Large UNIVERSITÀ DEGLI STUDI DI PADOVA}\\
        \vspace{0.2cm}                                                  %vspace serve per spaziare le righe
        {\Large \scshape Facoltà di Scienze MM. FF. NN.}\\
        \vspace{2mm}
        {\Large \scshape Corso di Laurea Magistrale in Informatica}\\
        \vspace{2mm}                                                      %aggiustatelo a vostro piacimento
        
        %\vspace{1cm}
        %{\LARGE Tesi di Laurea} \\
        \vspace{0.3cm}
        \includegraphics[width=6cm]{figure/unipd_logo}\\
        \vspace{2cm}
        
        %Tesi di Laurea Magistrale\\
        \vspace{2cm}
        {\LARGE \bfseries Performance Evaluation with Realistic Mobility of a File Sharing DTN Protocol} \\
        %{\LARGE \bfseries Performance Evaluation of a File Sharing DTN Protocol with Realistic Mobility} \\
        \vspace{2.5cm}
        %\includegraphics[width=7cm]{figure/alfresco_logo}\\
        %\vspace{3cm}
        %{\large }\\
        %\vspace{1cm}
    \end{center}

    \begin{tabular}{l}
        Author:\\
        Daniele Bonaldo\\
    \end{tabular}
    \hfill
    \begin{tabular}{l}
        Advisor:\\
        Prof. Claudio Enrico Palazzi\\
        \\                                      %lascia una riga bianca
        Co-Advisor:\\
        Ph.D. Armir Bujari 
    \end{tabular}
    \vfill
    \vspace{1cm}
        \begin{center}
            A.A. 2010-2011 \\
        \end{center}
\end{titlepage}

\clearpage{\pagestyle{empty}\cleardoublepage}


%%%%%%%%%%%%%%%%%%%%%%%%%%%%%%%%%%%%%%

%%%%%%%%%%%%%%%%%%%DEDICA%%%%%%%%%%%%%%%%%%%
%\begin{titlepage} 	%crea un ambiente libero da vincoli di margini e grandezza caratteri: si pu\`o modificare quello che si
%                 	%vuole, tanto fuori da questo ambiente tutto viene ristabilito
%\thispagestyle{empty} 		%elimina il numero della pagina
%\topmargin=6.5cm  			%imposta il margina superiore a 6.5cm
%\raggedleft               	%incolonna la scrittura a destra
%\large           			%aumenta la grandezza del carattere a 14pt
%\input{dedica}   			%richiama il file in cui c'è la copertina
%\newpage      				%va in una pagina nuova
%
%\end{titlepage}

\clearpage{\pagestyle{empty}\cleardoublepage} 	%apre una nuova pagina iniziando da quella destra ed altro (vedere manuale)

%%%%%%%%%%%%%%%%%%%%%%%%%%%%%%%%%%%%%%%%%%%%

\pagenumbering{roman}  		%serve per mettere i numeri romani in Introduzione, indice, elenco tabelle e figure

%%%%%%%%%%%%%%%INTRODUZIONE%%%%%%%%%%%%%%%%%%%

\rhead[\fancyplain{}{\bfseries                                               %imposta l'intestazione di pagina destra
Abstract}]{\fancyplain{}{\bfseries\thepage}}
\lhead[\fancyplain{}{\bfseries\thepage}]{\fancyplain{}{\bfseries             %imposta l'intestazione di pagina sinistra
Abstract}}
\chapter*{Abstract}
\label{abstract}
Qua ci va l'abstract
%\addcontentsline{toc}{chapter}{Abstract}                            %aggiunge la voce Introduzione nell'indice
\cleardoublepage

%%%%%%%%%%%%%%%%%%%%%%%%%%%%%%%%%%%%%%%%%%%%%%%%%%%%%%%%

%%%%%%%%%%%%%%Indice, elenco tabelle e figure%%%%%%%%%%%%%%%%%%%%%%%%

\rhead[\fancyplain{}{\bfseries\leftmark}]{\fancyplain{}{\bfseries\thepage}}
\lhead[\fancyplain{}{\bfseries\thepage}]{\fancyplain{}{\bfseries
INDICE}}
\tableofcontents                                                            %crea l'indice
\cleardoublepage



%%%%%%%%%%%%%%%%%%%%%%%%%%%%%%%%%%%%%%%%%%%%%%%%%%%%%%%%%%%%%%%%%%%%%%%%%%%%%%%%%%%%%%%%%%5

%%%%%%%%%%%%%%%%%%CAPITOLI%%%%%%%%%%%%%%%%%%%%%%%%%%%%%%

\lhead[\fancyplain{}{\bfseries\thepage}]{\fancyplain{}{\bfseries\rightmark}}    %imposta l'intestazione di pagina
\pagenumbering{arabic}                                                          %mette i numeri arabi


% this file is called up by thesis.tex
% content in this file will be fed into the main document

%: ----------------------- name of chapter  -------------------------
\chapter{Introduction}\label{introduzione} % top level followed by section, subsection


%: ----------------------- paths to graphics ------------------------

% change according to folder and file names
%\graphicspath{{2-Consorzi/images/}}


%: ----------------------- contents from here ------------------------
Since the origins of civilization, mankind has had the need to share informations between places very distant from each other. Communication means have evolved enormously through the centuries from the first letters carried by runners. Riding couriers moved mails through greater distances allowing at the same time a faster communication between far cities. The methods of communication kept changing and in the 1840's the electric telegraph was invented and messages were sent much quicker through the first invisible means of communication. Just twenty years later, the first transoceanic submarine cable had been laid, and it has been possible to instantaneously send messages between America and Europe. Next, the telephone was invented and people began to talk from great distances as if they were sitting in the same room. Finally, with the rise of the Internet a new world-scale communication became possible.
\\

In the last years the access to the internet has become available for a large number of people, using many different types of devices to connect to the network. Fifteen years ago internet access was a prerogative of wired connected computers and cellular phones were devices only able to make phone calls and send/receive text messages. Today, smartphones are becoming very popular and these devices evolved from simple mobile phones to devices able to browse into the internet, read emails and watching video streaming from the net.
\\

This new request for connectivity issued new problems, as new connections frontiers became available
: while the network's core is highly connected and well suited for routing via
conventional routing algorithms, the network‘s expanding frontiers have infrastructure
that suffers from intermittent connectivity and changes in topology that can be difficult
or impossible to predict.




parlare a voce come essere presenti di persona
i telefoni sono diventati mobili garantendo la possibilità di parlarsi a distanza ovunque si sia, anche in mobilità
i telefoni cellulari si sono poi evoluti, dai primi modelli che permettevano solo telefonate voce e al massimo messaggi di testo, fino ad apparecchi che permettono delle performances fino a qualche anno fa riservate a del personal computer.
dal navigare in internet, controllare le email, ascoltare musica in streaming
tutte queste nuove posibilità hanno però sollevato dei nuovi requisiti di connettività in situazioni difficili da coprire tramite le soluzioni abitualmente utilizzate in ambienti cablati, come è la struttura classica di internet
elevata mobilità e limited power



\section{Contributions}
In this project, we evaluate the performance of M2MShare, a protocol which implements DTN techniques in mobile phones world, in order to enable a peer-to-peer file sharing system between peers in the network using their mobility and opportunistic contacts among them. To do so we implement the protocol in the Oppostunistic Network Environment (ONE) simulator. This simulator is able to emulate human movement adopting several movement models in a map-based environment. To our judgement, our contributions are the following:
\begin{itemize}
\item implementing M2MShare and evaluating its behaviour using a simulator (the ONE simulator) able to emulate nodes movement in a realistic urban scenario.
\item evaluating the efficiency of the new paradigm created by M2MShare of use P2P solutions that matches file sharing with mobile users, allowing them to exchange files with each other.
\item evaluating the efficiency of using task delegations to dynamically establish forward routes along the destination path in the network.
\item evaluating the efficiency of the new file division strategy introduced with M2MShare 
\item enhancing M2MShare adding support to multi-hop delegations in order to further increase the search area for a single node to other disconnected overlay networks.
\item enhancing the ONE simulator adding some features to it and then making them available to the simulator users community.
\end{itemize}

\section{Document outline}
The remainder of this document is organized with the following chapters:
\begin{enumerate}
\setcounter{enumi}{1}
\item \textbf{Background:} in this chapter we describe some must-know technology, useful to the reader before to read the following chapters. We describe Delay Tolerant Networks (DTN) and Peer-to-Peer (P2P) systems.
\item \textbf{M2MShare:} this chapter describes M2MShare, the protocol object of the study and simulations of this thesis.
\item \textbf{Movement models:} in this chapter we give an overview about movement models used to simulate people movements. We describe the characteristics of several movement models, comparing them, and finally we describe the model we use for our simulations.
\item \textbf{The ONE simulator:} this chapter presents the ONE (Opportunistic Network Environment), the simulator we used for our evaluation of M2MShare.
\item \textbf{Implementation:} in this chapter we describe our implementation of M2MShare into the ONE and the improvements we did to the simulator to completely emulate the protocol behaviour.
\item \textbf{Simulations and results:} in this chapter we present the simulations we did. For each simulation we describe the settings, the protocol aspect evaluated, and finally we give the related results with a graphical representation.
\item \textbf{Conclusions:}  this chapter gives the concluding remarks, and proposes some related future works and enhancements. %It also contains the due acknowledgements.

\end{enumerate}
 
% ---------------------------------------------------------------------------
%: ----------------------- end of thesis sub-document ------------------------
% ---------------------------------------------------------------------------


% this file is called up by thesis.tex
% content in this file will be fed into the main document

%: ----------------------- name of chapter  -------------------------
\chapter{Background}\label{background} % top level followed by section, subsection


%: ----------------------- paths to graphics ------------------------

% change according to folder and file names
%\graphicspath{{2-Consorzi/images/}}


%: ----------------------- contents from here ------------------------
To fully understand how M2MShare works and what kind of simulator have to be used to emulate it, is worth to take a look at some technology backgrounds which can be useful to know before go further to the following of the thesis. In this chapter we describe what a DTN is, the notion of peer-to-peer and overlay networks.

\section{Delay/Disruption Tolerant Networks (DTN)}
The Internet is a connected network where internet protocols, most notably transmission control protocol/internet protocol (TCP/IP), are dependent upon (low) latencies of approximately milliseconds. This low latency, coupled with low bit error rates, allows TCP to reliably transmit and receive acknowledgements for messages traversing the terrestrial Internet. 
\\

A DTN is a network designed to operate effectively in highly-challenged environments where protocols adopted in connected networks (i.e. TCP/IP) fail. The "D" part in DTN acronym stands both for \textit{Delay} and for \textit{Disruption}. By delay we mean the end-to-end latency of data transmission. Some of those delays are inherent in the transmission medium, or the geometry of the system, but others are due to packets being temporarily stored on intermediate nodes. By disruption, we mean factors that cause connections to break down, or not be established, normally due to transient or quickly changing aspects of the system and/or its environment.
\\

There are some environments where low latency and end-to-end links are rarely available. One of the best examples of high latency with intermittent connectivity is that of space communications \cite{Burleigh2003365}. One-way trip times, at the speed of light, from the Earth to the Moon incurs a delay of 1.7 seconds; while one-way trip times to Mars incur a minimum delay of 8 minutes. The problem of latency for interplanetary links is exasperated with increased error rate due to solar radiation. In addition, the celestial bodies are in constant motion, which can block the required line-of-sight between transmit and receive antennas, resulting in links that at best are only intermittently connected. 
\\

DTNs need not be solely concerned with deep-space communications but can also be useful in terrestrial networks. In some environment, networks may be subjected to high disruption probability. One example is military application, where adopting DTNs allows the retrieval of critical information in mobile battlefield scenarios using only intermittently connected network communications. Another application, more peaceful, is the adoption of DTNs to overcome a major natural disaster. In such a situation terrestrial infrastructures may have been swept away and tolerant protocols must be used to coordinate rescue teams.  
\\
Networks adopted in these situations are characterized by:
\begin{itemize}
\item \textbf{Intermittent Connectivity:} if there is no end-to-end path between source and destination, end-to-end communication using the TCP/IP protocols does not work.
\item \textbf{Long or Variable Delay:} in addition to intermittent connectivity, long propagation delays between nodes and variable queuing delays at nodes contribute to end-to-end path delays that can defeat Internet protocols and applications that rely on quick return of acknowledgements or data.
\item \textbf{Asymmetric Data Rates:} the Internet supports moderate asymmetries of bidirectional data rate for users with cable TV or asymmetric DSL access. But if asymmetries are large, they defeat conversational protocols.
\item \textbf{High Error Rates:} bit errors on links require correction (which requires more bits and more processing) or retransmission of the entire packet (which results in more network traffic). For a given link-error rate, fewer retransmissions are needed for hop-by-hop than for end-to-end retransmission.
\end{itemize}

To overcome problems associated with all these factors, DTNs use a old yet affective method used in postal systems since ancient times. In this method, named \textit{Store-and-forwarding}, node physically delivers data to the destination moving from the source location to the destination of the recipient node (\figurename~\ref{fig:store-carry-forward}). Replication techniques can be adopted to increase the deliver ratio, copying the carried data and giving it to other nodes following a different physical path.

\begin{figure}[htpb]
  \begin{center}
    \includegraphics[scale=0.6]{2-background/img/store-and-forward.png}
    \caption{\textit{Store-and-forwarding} technique example}    
    \label{fig:store-carry-forward}
  \end{center}
\end{figure}

Messages are stored in storage mediums which can hold them for a long time period called persistent storage. This is another difference with internet protocols, where routers adopt very short-term storage provided by memory chips to store incoming packets. In Internet routers messages are queued only for a few milliseconds while they are waiting for their next hop. In DTNs routers need persistent storage for one or more of the following reasons:
\begin{itemize}
\item A communication link to the next hop may not be available for a long time and during this period message must be stored in the router.
\item There could be asymmetries in speed and reliability between nodes, so one node in a communicating pair may send or receive data much faster or more reliably than the other node.
\item A message, once transmitted, may need to be retransmitted if an error occurs at an upstream node or link, or if an upstream node declines acceptance of a forwarded message.
\end{itemize}


 
\section{Peer-to-Peer networks}
The most distinctive difference between Client/Server networking and Peer-to-Peer (P2P) networking is the role of each node in the network. In Peer-to-Peer networks nodes act as Servent, which is is an artificial word derived from the terms server and client. This term represent the capability of the nodes of a Peer-to-Peer network of acting at the same time as server as well as a client. In Client/Server networks each node can act as a server or as a client but cannot embrace both capabilities.
\\

In \cite{Aberer02anoverview} are identified three principles underlying P2P networks:
\begin{itemize}
\item  \textbf{Principle of sharing resources}: P2P systems involve an aspect of resource sharing, such as disk space network bandwidth or services. By sharing of resources applications can be realized which could not be set up by a single node. This was the driving motivation
behind a P2P system such as Napster.
\item \textbf{Principle of decentralization}: this is an immediate consequence of sharing
of resources. Parts of the system or even the whole system are no longer op-
erated centrally. Decentralization is in particular interesting in order to avoid
single-point-of failures or performance bottlenecks in the system. Examples
of fully decentralized systems are Gnutella and Freenet.
\item \textbf{Principle of self-organization}: when a P2P system becomes fully decen-
tralized then there exists no longer a node that can centrally coordinate it’s
activities or a database to store global information about the system centrally.
Therefore nodes have to self-organize themselves, based on whatever local in-
formation is available and interacting with locally reachable nodes (neighbors).
The global behaviour then emerges as the result of all the local behaviours
that occur.

\end{itemize}




\subsection{Peer-to-Peer networks as Overlay networks}
Peer-to-peer networks are typically constructed as overlay networks. These are computer networks which are built on top of other networks, adding an additional level of routing-logic. This is often done in the application layer, above transport and network layers, where maintenance and management algorithms operate. Overlays usually define some set of functions to forward overlay packets using this state. In order to implement such a higher-level of routing, many overlay networks define an additional level of logical routes above routes provided by the network layer. Each logical route can be composed by multiple lower-levels routes, i.e. routes composed by several IP-level routes.
\\

Overlay networks provide design flexibility and ease of deployment, at the cost of little inefficiency compared to a system implemented directly in coordination with the network layer. Also, designing a system using overlays helps to focus on the problem and functionalities to provide. While overlays are positioned at the application layer, it is possible to think to them as a distinct layer implementing higher-lever routing and transport services, while the complexities of network-level routing and transport are encapsulated in the lower layers.
\\

This allows to implement new services that are not yet available within the existing network. Some examples are the Distributed Hash Table (DHT) or the implementation of Virtual Private Networks (VPNs). 


%It also separates design concerns. Indeed, while overlays are positioned at the application layer, it is more fitting to think of them as a distinct layer implementing higher-lever routing and transport services. This separation of concerns partially decouples design, implementation, and optimization from the network and transport layer. The programmer is then free to focus solely on whatever problems are at hand, while the complexities of network-level routing and transport are encapsulated in the lower layers.
%\\

%The use of an overlay network in the design of new Internet services bypasses the need for strong social coordination during deployment. This has proven important in the continued evolution of the network, as individual autonomous systems and end-users can begin (or end) providing services in a piece-wise fashion. Multicast serves as a strong example of why such piece-wise role out is critical for eventual deployment.

\subsection{Peer-to-Peer architectures}
One ways to classify P2P networks is to consider the degree of centralization of the network and divides P2P networks in centralized or decentralized (or unstructured) networks.
\\

\paragraph{Centralized P2P networks} One of the most famous occurrence of the P2P paradigm was probably Napster\cite{Carlsson:2001:RFN:647728.734520}. This file-sharing system used an architecture in which a centralized entity provides a directory service to all participating peers, effectively forming a star network. All peers joining the system have to register their data with this centralized server. This allows other peers in the system to locate any data in the network by presence of a physically centralized directory. Only pointers to decentralized available peers are stored at the centralized server while the actual data is store in peers. When a peer found a pointer to relevant data in the directory, it could directly communicate with other peers that store the data in a decentralized manner, completely bypassing the centralized directory entity. Napster cannot be denoted as a pure P2P system because without central directory servers, single peers are not able to find resources shared by other peers.
\\

\paragraph{Decentralized P2P networks}
In decentralized (or unstructured) P2P architectures peers do not rely on any centralized entity to locate data items within the network. More specifically, in decentralized P2P networks, peers recursively forward received requests to neighbouring peers, in an attempt to find all relevant items in the network. This request forwarding is done using a broadcast technique know as message flooding. To prevent infinite loops and to control the number of messages generated by one single request, each message gets assigned a time-to-live (TTL) value. Each peer forwarding such a message decreases this value by one, and only messages with positive TTL values get forwarded. The main advantage of unstructured P2P networks is the fact that there is no need to maintain a network structure, as peers maintain only pointers to a limited number of direct neighbours. Also there is no need for a specific storage location for data items, as they can be located anywhere in the network.
\\

In \cite{Schollmeier:2001:DPN:882470.883282}, R. Schollmeier splits P2P networking
definition into two sub-definitions: \textit{pure} and \textit{hybrid} Peer-to-Peer networks.\\ 
The network is said pure if any peer in the underlying topology can be added and removed arbitrarily without having the network suffering any loss of network service. In other words, there is no special set of nodes that must be present for the service to work. \\
Peer-to-peer networks and algorithms in which an essential subset of peers exists are said to be hybrid networks. Hybrid networks typically have one or more strongly differentiated roles for various subsets of peers. Often role differentiation is related to the class of resource being shared. For example, one set of peers may handle storage, and another provide computational power. \\
%In other networks roles are differentiated across functionality, as is done in peer-to-peer file-sharing services such as Napster and BitTorrent, both of which separate indexing, and storage and download.
Pure peer-to-peer networks tend to generalize better, since they do not require any centralized, essential node to provide the related service.

%\section{Related works}
% ---------------------------------------------------------------------------
%: ----------------------- end of thesis sub-document ------------------------
% ---------------------------------------------------------------------------


% this file is called up by thesis.tex
% content in this file will be fed into the main document

%: ----------------------- name of chapter  -------------------------
\chapter{M2MShare}\label{m2mshare} % top level followed by section, subsection


%: ----------------------- paths to graphics ------------------------

% change according to folder and file names
%\graphicspath{{2-Consorzi/images/}}


%: ----------------------- contents from here ------------------------
La parte principale: si parla del protocollo
 
% ---------------------------------------------------------------------------
%: ----------------------- end of thesis sub-document ------------------------
% ---------------------------------------------------------------------------


% this file is called up by thesis.tex
% content in this file will be fed into the main document

%: ----------------------- name of chapter  -------------------------
\chapter{Modelli di movimento}\label{movimento} % top level followed by section, subsection


%: ----------------------- paths to graphics ------------------------

% change according to folder and file names
%\graphicspath{{2-Consorzi/images/}}


%: ----------------------- contents from here ------------------------
One important characteristic to achieve a good level of realism in DTN simulations is to simulate movement of nodes inside the simulated world. In Delay Tolerant Networks, movement of nodes is essential for the performance of the network, since end-to-end connectivity does not always exist and packets are delivered in a store and forward manner. For this reason is fundamental to simulate, with a good level of realism, movements behaviour of nodes inside the simulation.
\\

Capturing movement accurately in the real usage scenarios is also needed for a reliable assessment of a new protocol. Movement can be captured in simulations using real movement traces or synthetic traces generated by a movement model.

\section{Real user traces VS Synthetic models}
The most realistic user movement or contact patterns are the ones that happen in the real world. Therefore, several studies have focused on tracking user movement to be able to use the traces later directly for simulations or to learn more about which characteristics are common in real user behaviour.
Real movement traces have usually been obtained by analysing WLAN access point data or
having users carry around devices equipped with GPS modules. Contact traces have
mostly been collected from real world experiments where users have been carrying
around Bluetooth devices tracking other devices within range.

There are some problems about directly using the collected traces in simulations. Real world traces are usually from very specific environments like university campuses areas. These traces are not so generic to be used to simulate movement in a different environment, like a centre of a city, because behaviour and movement patterns of interested people are different. \\

Real trace are generally been obtained by analysing WLAN access point data or
having users carry around devices equipped with GPS modules. In the first case only certain areas have WLAN coverage and so movements outside access points communication range are not recorded. Using GPS modules limits mobility recording only to outdoor movements. 


When the locations of users are derived from access point data, the result is roughly building level granularity [16]. This is because users are not always connected to the closest access point and the movement between access points is difficult to capture. Most of the devices connecting to WLAN access points in these experiments are laptops and PDAs which are not always carried by the users and are not necessarily always turned on. Whether this characteristic is wanted in the simulations depends on what is modeled and simulated. Paper \cite{ImpactofHumanMobility} argues that the on/off times are an important characteristic of wireless users that needs to be taken into account and modeled for simulations. 

It is also practical to have a model with configurable parameters to work with. Sometimes a protocol or an application developer wants to test how the protocol or application performs when certain parameters change, to better be able to find out weaknesses and strengths. Sometimes researchers also need a very simple model to work with, to be able to use it in mathematical proofs for various theorems.


Synthetic models are often preferred since they are easier to work with than real user traces. Moreover, real user traces are rarely available for the environment to be modelled. Additionally, researchers want to do sensitivity analysis to find out how protocols and applications perform under different conditions. This is not possible with real user traces unless a parameterized model has been successfully extracted. 
There are two types of synthetic movement models that have been proposed for these
analyses — generic high level models that aim to produce movement accurate enough
with statistical measures, and models that describe incidental scenarios, hoping for a
more accurate depiction of single devices.
While efficient to use in simulations, the high level models, such as Random Waypoint
[18], often imply that the scenarios for which the protocols are simulated have huge
numbers of nodes, so that the relevant protocol features are given statistically realistic
distributions of events. For scenarios with few nodes, the differences between different
usage scenarios become more significant. Thus, movement models that depict more
precisely some specific types of movement are needed.
Traditionally, the approach to create a movement model has been to identify a certain
characteristic of mobility and to create a mathematical model describing the movement
at a high level. Such characteristics can be speed distribution, social relationships
between nodes or favorite locations nodes will visit. These types of movement have
very few details and the movement is homogeneous in the sense that every node is
moving according to the same rule.

Before we go any further, we introduce two new terms: locality and heterogeneity. By
locality we mean the tendency of nodes spending most of their time within a small area.
Thus, a movement model where each node's movement is restricted to a small area has
high locality. By heterogeneity, we mean different movement patterns and properties
between nodes. Measuring heterogeneity in a specific context is not straightforward,
but in our experiments we will later on talk about differences in contact patterns in
terms of how often a node encounters new nodes compared to the fraction of earlier
encountered nodes.

Hsu and Helmy [26] show by studying real user traces that nodes are very often turned
on/off and only visit a small portion of the WLAN access points in campus areas.
Moreover, they find that node mobility while using network is very low and one node
only meets a small portion of all other nodes in the area. These types of characteristics
are usually not captured in movement models. Furthermore, they reveal repetitive
patterns with a period of one day and heterogeneity among nodes. Although
heterogeneity and repetitiveness has been modeled, most simple movement models do
not. According to Hsu and Helmy, the biggest issue with most synthetic models is that
they are not capturing such characteristics as heterogeneous behavior, switching
devices on/off or relationships between users.
We noticed that even though most of the known features of movement have been
modeled, no model exist where all or many of the features have been combined. Our
approach is to combine these different elements to create a new movement model.



\section{Random Map-Based Movement}
The simplest map-based mobility model is Random Map-Based Movement (MBM). Nodes adopting this model moves randomly but always in streets described in the map. This is done walking from one map node, i.e. an address on a street or a crossroad, to the other by always randomly selecting one of the directly connected map nodes. Result of this behaviour is a random movement in which there are more contacts between nodes than in Random Walk model. This is due to the restriction to node movement to follow streets described in the map.


\section{Shortest Path Map-Based Movement}
Shortest Path Map-Based Movement (SPMBM) is another mobility model that uses a map-described environment to restrict node movement. With this model, nodes chooses their destination randomly inside the map, then calculate the shortest path to reach it and finally walk along this path. Dijkstra's algorithm is used to calculate the path. Destinations can be chosen in a totally random way or from a set of Points of Interests (POI). This can be useful to emulate interesting places like restaurants, monuments or shops. This model adds one level of realism, compared to MBM, 

\section{Routed Map-Based Movement}
Un modello che invece della casualità nel movimento utilizza dei percorsi predeterminati è il Routed Map-Based Movement (RMBM). I nodi che adottano questo modello si muovono lungo rotte predefinite, per tutta la durata della simulazione, rendendo RMBM utile per rappresentare degli spostamenti ripetitivi, come ad esempio quelli di autobus, tram o treni.

\section{Working Day Movement Model}
\label{descrWDM}
I modelli finora descritti sono senz'altro di semplice comprensione e realizzazione, oltre ad essere molto efficienti per quanto riguarda le prestazioni, ma non forniscono una realistica rappresentazione del movimento umano, soprattutto per quanto riguarda i valori di inter-contact e contact time.
\\
Un modello che genera dei valori più realistici per questi parametri, rappresentando più realisticamente quindi il movimento umano, è il Working Day Movement Model (WDM). Come il nome può fare intuire, questo modello simula gli spostamenti compiuti da una persona durante una tipica giornata lavorativa e in \cite{articoloWdm}, dove è descritto il modello, è evidenziato anche come i valori generati seguano realisticamente quelli trovati utilizzando dati di spostamento provenienti da tracce reali. 
\\
Una giornata simulata comprende le seguenti attività principali svolte dai vari nodi:
\begin{itemize}
\item dormire a casa
\item lavorare in ufficio
\item uscire alla sera con gli amici
\end{itemize}

Ovviamente le attività potrebbero cambiare enormemente a seconda dello stile di vita e del lavoro svolto dalle singole persone, ma queste tre attività sono le più comuni e possono essere associate alla tipica giornata di una gran quantità di persone.
La ripetitività giornaliera delle azioni e il fatto di svolgerle in luoghi comuni a più persone permette la formazione spontanea di comunità: persone che vivono e dormono nella stessa casa formeranno una famiglia, più persone che lavorano nello stesso ufficio agli stessi orari saranno colleghi di lavoro, mentre degli amici si possono trovare ad orari comuni alla sera per uscire assieme.
La creazione di queste comunità non viene mostrata da modelli più semplici quali RMBM o SPMBM.
\\

Per la simulazione delle attività giornaliere, WDM utilizza dei sottomodelli dedicati, oltre a dei sottomodelli preposti a simulare gli spostamenti fra un'attività e l'altra. Una persona si potrà quindi spostare a piedi, in auto o utilizzando i mezzi pubblici, a seconda della propria disponibilità e convenienza. Il fatto di muoversi da soli o in gruppo (prendendo lo stesso bus o camminando assieme la sera) permette di avere dei comportamenti eterogenei e quindi migliorare ulteriormente la realisticità degli spostamenti compiuti dai vari nodi.
\\

\subsection{Esempio di giornata}
Durante una tipica giornata il punto di partenza per ogni nodo è la propria abitazione. Ogni nodo ha un orario di sveglia assegnato, generato utilizzando una distribuzione normale con media pari a 0 e deviazione standard impostabile in fase di configurazione, che indica l'orario in cui la persona uscirà di casa. Il valore viene generato per ogni ad inizio simulazione, rimarrà lo stesso per tutti i giorni successivi e la differenza fra i valori di vari nodi sta ad indicare la differenza fra i diversi stili di vita nella vita reale (ad esempio una persona che impiega pochi minuti a prepararsi la mattina rispetto a chi impiega ore anche solo per fare colazione).
\\ 

Una volta usciti di casa, i vari nodi si dirigono al lavoro utilizzando l'auto (se disponibile) o a piedi oppure utilizzando i mezzi pubblici, a seconda di quale sia il metodo più conveniente. Conseguentemente alla scelta del mezzo di trasporto viene utilizzato il corrispondente sottomodello.
\\

Una volta raggiunto il luogo di lavoro, la persona ci resta per la durata della sua giornata lavorativa e quindi decide, con una determinata probabilità, se tornare direttamente a casa o spostarsi per un'attività serale. Anche in questo caso gli spostamenti vengono gestiti utilizzando i corrispondenti sottomodelli.


\subsection{Home Activity Submodel}
Ogni nodo ha una posizione impostata come Home Location, che viene utilizzata come punto di partenza alla mattina e punto di ritorno alla sera: una volta tornata a casa una persona si muove per una breve distanza e poi resta ferma fino all'orario di risveglio, la mattina successiva. Questo comportamento non è un errore, ma simula il fatto di lasciare il telefono su di un tavolo o in carica fino al momento di uscire nuovamente di casa, mentre la persona svolge le normali attività domestiche come mangiare, guardare la TV o dormire.

\subsection{Office Activity Submodel}
Il sottomodello relativo all'attività lavorativa è un modello bidimensionale che simula il comportamento di una persona all'interno di un ufficio, in cui è posizionata la propria scrivania e dalla quale ogni tanto si alza per partecipare ad una riunione, parlare con un collega o, perché no, per una pausa caffè. Durante tutti questi momenti, come è facile intuire, è possibile che il nodo entri in contatto con nodi relativi ad altri colleghi di lavoro.
\\

L'ufficio è descritto come un'unica stanza con pianta rettangolare, in cui l'unico punto di ingresso, la porta, è l'angolo in alto a sinistra e ogni persona che vi lavora ha una scrivania posizionata in un determinato punto. Non viene simulata la presenza di muri all'interno della stanza, che quindi verrà descritta come un luogo più grande del normale, in modo da simulare il tempo impiegato per superare ostacoli, nel movimento dalla scrivania ad una destinazione interna all'ufficio.
\\

Una volta entrato, l'impiegato si muove subito camminando verso la propria scrivania, dove rimane per un periodo di tempo casuale, generato utilizzando una distribuzione di Pareto. Passato questo tempo il nodo sceglie una destinazione casuale all'interno dell'ufficio, cammina fino a raggiungerla e quindi attende per un periodo di tempo casuale generato utilizzando la stessa distribuzione di Pareto prima di tornare alla propria scrivania. La ripetizione del movimento dalla scrivania ad una posizione casuale interna all'ufficio continua fino al termine della giornata lavorativa, che ha una durata impostabile in fase di configurazione.
\\

I parametri della distribuzione possono essere impostati per ogni gruppo di nodi, in modo da simulare diversi tipi di attività all'interno del luogo di lavoro, dall'insegnante che ogni ora si deve spostare in un'aula diversa, ad un commesso che non lascia mai la propria postazione per tutto l'orario lavorativo.


\subsection{Evening Activity Submodel}
Il sottomodello Evening Activity simula attività che possono essere svolte dopo lavoro, nel tardo pomeriggio - sera, come andare a fare shopping, in un bar o a mangiare in una pizzeria o ristorante. Tali attività vengono svolte in gruppo e con una probabilità configurabile, che determina se la persona torna o meno subito a casa dopo il lavoro.
\\

Al termine della giornata lavorativa, il nodo si sposta verso il proprio luogo d'incontro preferito, che è una posizione impostata all'inizio della simulazione. Una volta arrivato attende che lo raggiungano un numero di persone sufficientemente elevato per formare un gruppo e cominciare quindi l'attività. Il numero massimo e minimo di persone che possono formare un gruppo è configurabile e quando tutti i gruppi per un determinato punto di ritrovo sono al completo ne viene creato un altro.
\\

Una volta che tutti i componenti del gruppo legato all'attività sono arrivati, camminano assieme per una breve distanza verso una destinazione scelta casualmente e quindi si fermano per un tempo più lungo, generato casualmente all'interno di valori preimpostati. Una volta terminata questa pausa (finita la cena, lo shopping o la visione di un film al cinema), le varie persone si separano e tornano verso casa.

\subsection{Transport Activity Submodel}
Il Transport Activity Submodel è il sottomodello incaricato di gestire gli spostamenti dei nodi fra le diverse attività.
\\

All'inizio della simulazione ad ogni nodo viene assegnata un'auto con una probabilità configurabile. Le persone che la possiedono la utilizzeranno per tutti gli spostamenti, mentre chi ne è sprovvisto si muoverà a piedi o utilizzando un mezzo pubblico. L'eterogeneità di mezzi di trasporto utilizzati permette di simulare realisticamente i movimenti di diverse tipologie di persone ed inoltre ha impatto anche sul protocollo di routing utilizzato, in quanto nodi che si muovono utilizzando mezzi propri si sposteranno più velocemente, permettendo così il trasporto più rapido di pacchetti per lunghe distanze.
\\

A seconda del mezzo di trasporto utilizzato, quindi, il Transport Activity Submodel si rifà a tre sottomodelli distinti:
\\

\begin{description}
\item [Walking Submodel]: i nodi che non possiedono un'auto si muovono camminando lungo le strade ad una velocità costante, utilizzando l'algoritmo di Dijkstra per provare il percorso più breve dalla posizione corrente alla destinazione.
\item [Car Submodel]: i nodi che possiedono un'auto muovono più velocemente dei pedoni, durante le transizioni fra attività, ma si muovono come gli altri nodi all'interno di una singola attività. Non vengono considerati traffico e semafori durante la guida e ogni auto può portare una sola persona (il car sharing non ha ancora avuto successo nel mondo simulato).
\item [Bus Submodel]: nella città possono essere presenti più linee di trasporti pubblici (tram, autobus, funivie), ognuna delle quali viene percorsa da più mezzi ad orari predefiniti. Ogni mezzo pubblico può trasportare più persone.
\end{description}


Ogni persona che non possiede un'auto conosce una linea di mezzi pubblici e può utilizzare qualunque mezzo appartenente a quella linea. Il fatto di prendere il mezzo pubblico o di camminare dipende da un confronto di distanze euclidee: la distanza fra il luogo di partenza e quello di destinazione oppure quella fra il luogo di partenza e la fermata più vicina sommata alla distanza fra la destinazione e la fermata più vicina alla destinazione. Nel caso sia minore la prima allora il nodo camminerà fino alla destinazione, altrimenti utilizzerà i mezzi pubblici. Per fare ciò camminerà fino alla fermata più vicina, utilizzando il Walking Submodel, attenderà il primo mezzo che passerà per quella linea nella direzione corretta, utilizzerà il Bus Submodel fino alla fermata più vicina alla destinazione e quindi tornerà ad utilizzare il Walking Submodel camminando fino al punto di arrivo.
 
% ---------------------------------------------------------------------------
%: ----------------------- end of thesis sub-document ------------------------
% ---------------------------------------------------------------------------


% this file is called up by thesis.tex
% content in this file will be fed into the main document

%: ----------------------- name of chapter  -------------------------
\chapter{The ONE simulator}\label{simulatore} % top level followed by section, subsection


%: ----------------------- paths to graphics ------------------------

% change according to folder and file names
\graphicspath{{2-simulatore/img/}}


%: ----------------------- contents from here ------------------------
The simulation environment chosen for our simulations is The ONE (Opportunistic Network Environment) version 1.4.1\footnote{The ONE simulator \href{http://www.netlab.tkk.fi/tutkimus/dtn/theone/}{http://www.netlab.tkk.fi/tutkimus/dtn/theone/}}, described in \cite{articoloONE}. This simulation environment written in Java is completely configurable and is able to simulate completely the behaviour of nodes in the simulations, including movement, connections and to give a visual feedback about all these factors using its GUI. A more interesting feature, for our purposes, is the capability of emulate message routing using different routing protocols.
\\

To emulate nodes movement, the ONE can take in input traces from real-world measurements, from external mobility generators or it can generate movement patterns using synthetic mobility model generators. Differences between several available mobility are explained in Section \ref{movimento}. Either mobility models that routing protocols are managed like independent modules, which are dynamically loaded depending on what set in configuration files. This allow a relatively easy implementation of new mobility models and routing protocols in the simulator.
\\

The simulator finally allow to save data of interest from the completed simulations into report files. This reports are created using modules dynamically loaded in the same way to what happens with movement and routing modules.
\\

\section{Configuration}
\label{configurazioneONE}
A single simulation is set, before running, using configuration files which describe the simulated environment, from the simulation length to number of nodes emulating people in the simulated environment. Configuration files are text-based files that contain parameters about the simulation, user interface, event generation, and reporting modules. All these parameters are loaded before the beginning of simulations and are used to adjust details about the behaviour of loaded modules.
\\

Many of the simulation parameters are configurable sepa-
rately for each node group but groups can also share a set of param-
eters and only alter the parameters that are specific for the group.
The configuration system also allows defining of an array of val-
ues for each parameter hence enabling easy sensitivity analysis: in
batch runs, a different value is chosen for each run so that large
amounts of permutations are explored.

Inside configuration files, parameters are saved as key-value pairs and syntax for most of the
variables is:

\begin{center}
\textit{Namespace.key = value}
\end{center}

Namespace defines the part of the simulation environment where the setting has effect on. Many, but not all, namespaces are equal to the class name where they are read. This convention is especially followed by movement models, report modules and routing modules, so also created new modules should follow it.
\\

To make human-readability and configuration easier, numerical values can use suffixes kilo (k), mega (M) o giga (G), with \textquotedblleft .\textquotedblright as decimal separator. Boolean parameters accept \textquotedblleft true\textquotedblright
 or \textquotedblleft 1\textquotedblright , \textquotedblleft false\textquotedblright or \textquotedblleft 0 \textquotedblright .
\\
 
Comments can be inserted in setting files can contain comments too with \textquotedblleft
\#\textquotedblright character. Rest of the line is skipped when the settings are read.
\\

Every simulation can be set using several configuration files, to divide parameters into different categories, e.g. a file can contain scenario parameters, like maps, streets and districts, another file can contain parameters used to configure nodes, another file can contain reports configuration and so on. First configuration file read, if exists, is always \textquotedblleft default\_settings.txt\textquotedblright. Other configuration files given as parameter can define more settings or override some (or even all) settings in the previous files. The idea is that you can define in the earlier files all the settings that are common for
all the simulations and run different simulations changing parameters only in latests configuration files.
\\

The basic agents in the simulator are called nodes. A node models a mobile endpoint capable of acting as a store-carry-forward router (e.g., a pedestrian, car or tram with the required hardware). Nodes in the simulation world are divided into groups, each one configured with different capabilities. Inside a group, every node has the same characteristics, which are radio interface, persistent storage, movement, energy consumption and message routing protocol. It is possible to configure some of these capabilities as common for all groups, avoiding to set them individually for each group. Changing some configuration value between a group and another, allow to get heterogeneity between the behaviour of nodes in simulated scenario.
\\

In configuration files is also possible to provide array of settings for every parameter. This allows to run large amounts of different simulations using only single configuration file. Every simulation will be different from the previous and this is very useful to gather reports about different aspects in a common scenario.
\\ Syntax for these configuration parameters is

\begin{center}
\textit{Namespace.key = [run1_value; run2_value; run3_value; etc]}
\end{center}

Some parameters, finally, accept a file path as value and this can be either an absolute or relative path. These parameters are used for maps, nodes paths or events to be loaded by event generators modules.

\section{Visualization}

La principale modalità di visualizzazione fornita da ONE è quella tramite la GUI, che permette di seguire in tempo reale l'avanzamento della simulazione. Nella finestra principale è possibile osservare i movimenti dei vari nodi e, selezionandone uno specifico, ottenere informazioni riguardo le connessioni attive, i messaggi trasportati e altri dettagli. E' disponibile inoltre un riquadro in cui viene costantemente aggiornato un log di eventi generati durante la simulazione che possono essere filtrati a seconda di ciò che più interessa (ad esempio visualizzare solo le nuove connessioni o gli scambi di messaggi).
\\
\figuremacro{Schermata-ONE}{Schermata Principale}{La schermata principale del simulatore ONE}{}

Nel caso di modelli di movimento basati su una mappa, selezionando un nodo sarà possibile vedere il percorso seguito, la destinazione da raggiungere e ottenere informazioni avanzate riguardanti lo stato di quel nodo (connessioni, messaggi trasportati, ecc), come si può vedere in Figura \ref{Routing-Info}. La visualizzazione è personalizzabile zoomando, modificando la velocità di avanzamento e anche inserendo nello sfondo un'immagine, come ad esempio una carta stradale o una fotografia satellitare della zona interessata.
\\
\figuremacro{Routing-Info}{Routing Info}{Un esempio di finestra in cui sono presenti i dettagli relativi allo stato di un nodo}{}


L'altra modalità di seguire l'avanzamento di una simulazione è la lettura dei reports generati dai vari moduli durante l'esecuzione. Come i modelli di movimento e i protocolli di routing, i generatori di reports sono caricati dinamicamente all'avvio della simulazione, a seconda di ciò che è stato impostato nei files di configurazione.
Questa modalità di visualizzazione è particolarmente utile quando non si utilizza la GUI, ma si eseguono più simulazioni in batch, ottenendo quindi alla fine i report con i dati raccolti durante le varie simulazioni.
\\

La modalità batch è indicata nel caso si debbano eseguire più simulazioni in serie senza essere interessati a seguirne l'avanzamento in modalità grafica ma piuttosto valutandone i risultati una volta completate. In questo caso si dimostra molto utile la possibilità di specificare array di valori per i parametri di configurazione, in modo da poter programmare in anticipo le differenze fra le varie simulazioni della serie. Una volta avviato quindi il batch di simulazioni, ONE si occuperà si eseguirle in sequenza e alla massima velocità permessa della macchina in uso e salverà i reports generati in più files impostati durante la configurazione.

\section{Reports}
Il simulatore può gestire la generazione di più reports relativi alla simulazione in esecuzione. Questi report vengono creati da dei moduli attivati in fase di configurazione e consistono generalmente in files di testo in cui vengono salvati i dati e statistiche che verranno poi analizzati a simulazione terminata. Nella versione 1.4.1 di ONE, il sistema permette di generare reports relativi a:
\begin{description}
\item[messaggi,] che includono numero di messaggi creati, scambiati, scaduti, ecc
\item[contatti,] in cui viene indicato il contact e l'inter-contact time fra i vari nodi, oltre al totale dei contatti durante la simulazione.
\item[connessioni,] che descrivono l'alternarsi di stato delle connessioni 
\end{description}

Come per le altre parti di cui ONE è composto, anche i generatori di reports vengono gestiti come moduli, ed è quindi possibile aggiungerne di nuovi a seconda delle esigenze e dei dati da raccogliere nella singola simulazione.


\section{Esecuzione}
\label{esecuzioneONE}
Vale la pena di soffermarsi sull'esecuzione di una singola simulazione, vedendo quindi quali sono i vari passi che vengono eseguiti.
\\

La prima azione svolta dal simulatore è quella di caricare le impostazioni dai vari files di configurazione, la cui posizione viene passata per parametro al momento dell'esecuzione del simulatore. Man mano che un nuovo file di configurazione viene letto, i valori dei parametri in esso contenuto vanno ad impostare il valore di alcune variabili dell'ambiente di simulazione, sovrascrivendone anche il valore nel caso fossero già state impostate.
\\

Una volta caricate le impostazioni relative alla simulazione, viene creato lo Scenario. Questo contiene al suo interno tutti gli elementi attivi durante la simulazione (come i nodi, i generatori di reports e quelli di messaggi), così come quelli passivi (ad esempio le mappe che compongono il mondo simulato). In questa fase vengono quindi creati tutti i nodi partecipanti alla simulazione, ognuno dotato di un proprio modello di movimento, un router configurato secondo le caratteristiche del gruppo di nodi e una serie di \textit{listener} per la cattura di eventi e la successiva generazione di reports.
\\

Quando tutte le entità necessarie all'esecuzione sono stati create e configurate, si passa all'esecuzione vera e propria.
Questa consiste nel ripetere l'aggiornamento dello stato del mondo, chiamando un metodo \textit{update()}, e incrementare il valore del tempo simulato fino al raggiungimento di un tempo impostato come fine simulazione. L'incremento temporale che viene effettuato ad ogni aggiornamento è impostato nei files di configurazione (con il parametro \textit{Scenario.updateInterval}), espresso in secondi, ed influenza i vari modelli di movimento dei nodi. La prima operazione svolta durante l'\textit{update()} del mondo simulato è lo spostamento dei vari nodi, che avviene a seconda del modello di movimento adottato dal nodo e dell'incremento temporale applicato. Ad esempio un nodo che simula un automobilista si sposterà maggiormente rispetto ad uno che simula un pedone, a parità di intervallo di tempo simulato. Come vedremo nella sezione \ref{movimento} relativa ai modelli di movimento, questi possono essere anche molto complessi e simulare diversi comportamenti a seconda delle configurazioni adottate.
\\

Una volta effettuato il movimento, per ogni nodo viene aggiornato lo stato delle connessioni e del router simulato. Per ogni interfaccia di rete disponibile viene quindi aggiornato lo stato a seconda che lo spostamento abbia comportato una caduta della connessione o abbia permesso di entrare nel raggio di comunicazione di un interfaccia di rete relativa ad un altro nodo. Ogni qualvolta lo stato di una connessione cambia, vengono avvisati i \textit{listeners} interessati, per la generazione di report, e viene aggiornata la visualizzazione grafica della connessione, se attiva, come si può vedere, ad esempio, in Figura \ref{connessioni}.
\\
\figuremacro{connessioni}{connessioni}{Un esempio di connessioni tra nodi tratto dalla finestra principale di ONE. Nella parte sinistra dell'immagine si può notare la connessione attiva, mentre nella parte a destra i due nodi non sono più l'uno nel raggio di comunicazione dell'altro, indicato dal cerchio verde attorno al nodo.}{}

L'ultima parte dell'aggiornamento relativo allo stato di un nodo riguarda l'aggiornamento del router. Questo è fortemente dipendente dal protocollo di routing implementato ed è proprio nell'esecuzione del metodo \textit{update()} relativo al router che si svolgono le azioni caratteristiche di un protocollo rispetto ad un altro.

\section{Casualità}
come viene introdotta della casualità nelle varie simulazioni

\section{Limitazioni}
\label{limitazioniONE}
non più in basso del livello di routing
simulazione temporale discreta
mancanza di simulazione di un file system

\section{The map}
\label{mappaONE}
qua ci va la descrizione della mappa con la divisione in distretti e magari un'immagine delle rotte dei bus e del modello di movimento
% ---------------------------------------------------------------------------
%: ----------------------- end of thesis sub-document ------------------------
% ---------------------------------------------------------------------------


% this file is called up by thesis.tex
% content in this file will be fed into the main document

%: ----------------------- name of chapter  -------------------------
\chapter{Implementation}\label{implementazione} % top level followed by section, subsection


%: ----------------------- paths to graphics ------------------------

% change according to folder and file names
\graphicspath{{6-implementazione/img/}}


%: ----------------------- contents from here ------------------------
In this chapter we describe our implementation of M2MShare in the ONE simulator, which is described in Chapter \ref{simulatore}. We illustrate some features we added to the simulator, in order to completely emulate M2MShare behaviour and to gather reports data from the simulations we executed.
  

\section{ONE additions}
As we saw in Chapter \ref{m2mshare}, M2MShare is a peer-to-peer protocol which allows the automatic exchange of multimedia content among mobile devices. The ONE is the state-of-the-art regarding movements and connections simulation, especially between mobile nodes, but it has some limitations. Before we could completely implement M2MShare in the simulator, we modified the ONE adding some fundamental features. These allow us to simulate the behaviour of a peer-to-peer protocol with file exchange between network nodes.


\subsection{File Management}
The first step is to add the capability of manage files to nodes. To this purpose, we added a file system to every node, in which to save files to share with other nodes using a P2P application. 
%Node's capabilities are described in class named \textit{DTNHost}


\paragraph{DTNFile}
Every simulated file is an instance of class \textit{DTNFile}, which includes information about the described file:

\begin{itemize}
\item file name
\item file hash value
\item file length, in bytes
\end{itemize}

We do not specify keywords related to that file, which could be useful in implementing a keyword-based search procedure. We did so, because the purpose of our simulations was to emulate and evaluate the behaviour of M2MShare in the second phase of the protocol. In this phase, the user already knows what file it is looking for, so it asks the system to find and download back it, without needing of previous keyword queries. For this purpose a file hash value is sufficient to unambiguously identify the file in the overlay network.


\paragraph{DTNFileSystem}
\textit{DTNFile}s shared by a node are included in an entity called \textit{DTNFileSystem}. As its name suggests, it emulates a mobile device's file system, from an high-level prospective. It has a simple implementation, with a single-level directory structure and functions which allow inserting, deleting and reading files from the file system. To retrieve a file, its hash value is used as search criteria.
\\

The file system presence in a simulated node is optional. For instance, a node emulating a bus do not need a file system. We added new parameters, settable in configuration files, which permit the specification of more options related to file management for a single simulation and increase simulation flexibility: 

\begin{description}
\item[Scenario.simulateFiles] is a boolean parameter that indicates whether or not to simulate the files management in mobile nodes. It applies to the whole simulation.
\item[Group.fileCapability] is a boolean parameter that indicates if nodes belonging to a group can use a file system to store files.
\end{description}

\paragraph{DTNFileGenerator}
\label{fileGeneratorImplementazione}
In a P2P network, shared files are distributed among peers and a user can search between them for the file it is looking for. To emulate the initial files distribution, we implemented a generator which creates and distributes files between nodes at the beginning of the simulation. This generator is initialized using configuration files. Each of them can include several \textit{DTNFileCreationRequest}. These requests describe the files to create and how to distribute them among simulated nodes. During simulation's configuration is possible to set:

\begin{itemize}
\item type of file distribution 
\item name of the \textit{DTNFile}
\item length in bytes
\item number of copies to distribute in the entire network
\end{itemize}

Files are distributed only among nodes which have a file system, that is nodes in groups with \textit{fileCapability} parameter set to \textit{true}. Files distribution can be chosen from the following:

\begin{description}
\item[A] (all): file is distributed among all nodes of the simulation with equal probability
\item[G] (groups): file is only distributed among nodes belonging to specified groups
\item[P] (percent): file is distributed among all nodes of the simulation with probability set in the request
\item[H] (hosts): file is only distributed among specified nodes  
\end{description}

The following are some example of input files used to create and distribute \textit{DTNFiles}, using the \textit{DTNFileGenerator}.

\begin{center}
%\framebox[8cm]{\textbf{A	mySong.mp3	3.5M	50}}
\textbf{A	mySong.mp3	3.5M	50}
\end{center}
Indicates to create a \textit{DTNFile} with size 3,5 MB named "mySong.mp3". The file is copied 50 times and distributed with equal probability among all users with \textit{fileCapability = true}.
\\

\begin{center}
\textbf{P	aPhoto.jpg	5M	25}
\end{center}
Indicates to create a \textit{DTNFile} with size 5.0 MB named "aPhoto.jpg". The file is distributed to 25\% of nodes \textit{fileCapability = true}.
\\

\begin{center}
\textbf{G	aPhoto.jpg	2.4M	25	6	8	10}
\end{center}
Indicates to create a \textit{DTNFile} with size 2,4 MB named "aPhoto.jpg". The file is copied 25 times and distributed with equal probability among all users belonging to groups 6, 8 and 10
\\

\begin{center}
\textbf{H	ebook.pdf	650k	42	43	44}
\end{center}
Indicates to create a \textit{DTNFile} with size 650 kB named "ebook.pdf". The file is distributed among nodes with address 42, 43 e 44, one copy for every node.
\\

We saw in Section \ref{randomness} that an important characteristic of simulations is repeatability. This is fundamental to verify correctness of data and results of completed simulations, or to repeat the simulation changing only few parameters over the total configuration, leaving constant the simulated world description. This applies also to randomness included in \textit{DTNFileGenerator}.\\
To guarantee that repeating one simulation with the same values, same results are achieved, we use a strategy similar to the one adopted by movement models. \textit{DTNFileGenerator} is initialized with a configuration parameter related to \textit{random number generator} seed. If a simulation is repeated several times with the same random seed, files are distributed among the same subset of nodes, at the beginning of the simulation.


%Per garantire che, effettuando più simulazioni utilizzando la medesima configurazione, i files vengano distribuiti agli stessi nodi, è stata adottata la stessa tecnica utilizzata dai modelli di movimento che estendono la classe \textit{MovementModel}. In quel caso i moduli relativi al movimento vengono inizializzati utilizzando un parametro di configurazione relativo al seed per il \textit{random number generator}. Nel caso di valori di configurazione e seed iniziale mantenuti immutati, i nodi si muovono nello stesso modo anche ripetendo più volte la simulazione: vengono infatti generati gli stessi valori relativi a destinazioni, velocità  e tempi di attesa per i nodi.
%\\
%If the seed and all the movement model related settings are kept the same, all nodes should move the same way in different simulations (same destinations, speed and wait time values are used).


\paragraph{FileRequest}
A \textit{FileRequest} represents the user request for a particular file. In every instance of this class are included:
\begin{itemize}
\item \textbf{fromAddr}: the address of the node operated by the user looking for the file
\item \textbf{filename}: the name of the file searched
\item \textbf{creationTime}: the simulated time when the request s submitted by the user 
\end{itemize}
All \textit{FileRequests} are read by the \textit{M2MShareFileRequestReader} at the beginning of every simulation, then during the simulation, when simulated time become equal to the \textit{creationTime} of the \textit{FileRequests}, that is inserted into the correct queue of the node corresponding to the address included in the request.

\paragraph{M2MShareFileRequestReader}
It is the entity responsible for reading the \textit{FileRequests} at the beginning and to dispatch the requests during the simulation, when the correct time comes. It implements the interface \textit{EventQueue}, and so it provides the method \textit{nextEventsTime()}, which returns the instant when the next request will be ready to be inserted in the corresponding node, and the method \textit{nextEvent()}, which returns the next request that will become active.


\subsection{GUI}
The GUI of the ONE simulator has been modified to reflect the addition of file simulation support to the simulator. The detail window related to a node (\figurename~\ref{implRouting-Info}), shown using the button \textquotedblleft Routing info\textquotedblright \ now contains informations about the file system of the node, including all the complete files owned, the number of active tasks and the percent of data already downloaded for each task. 
\begin{figure}[htpb]
  \begin{center}
    \includegraphics[scale=0.6]{6-implementazione/img/Routing-Info.png}
    \caption{An example of window with a detailed view of a node state}    
    \label{implRouting-Info}
  \end{center}
\end{figure}

\subsection{Reports}
\label{descrReports}
A fundamental feature related to simulations is the ability to gather data from different aspects of the simulated world, like positions of nodes, communications, data transfers and so on. To achieve this goal in the ONE are used some special modules named \textit{Report Generators} which works with some related Listeners to catch the interested events in the simulated scenario and save them to some report files.
The main objectives of our simulations was to gather information about the efficiency of delegations and file division strategies, and to do so we collected data about several aspects for every simulation. We will now describe the report modules created to the gathering of data during the simulations, specifying the type and mode of data collection used.

\paragraph{FileGatheringLog}
The first report module is a log report module, that stores the selected events in a list, saving them to file, one event per line. This module is responsible for listening for the following events:
\begin{itemize}
\item \textbf{VirtualFile creation:} a new \textit{VirtualFile} has been created due to the execution of a \textit{FileRequest} in a node. It emulates a user requesting a new file to be found and downloaded.
\item \textbf{DTNPendingDownload creation:} a new \textit{DTNPendingDownload} has been created in a servant peer thanks to the delegation of a task from a client node.
\item \textbf{DTNPendingDownload completion:} a \textit{DTNPendingDownload} has been completed in a servant peer (the servant has downloaded all the requested file intervals subject of the delegated task). Consequently a \textit{DTNDownloadFWD} has been created and queued in the servant.
\item \textbf{DTNPendingDownload expiration:} a \textit{DTNPendingDownload} has expired in a servant peer before being completed (the servant has not downloaded all the requested file intervals subject of the delegated task).
\item \textbf{DTNDownloadFWD expiration:} a \textit{DTNDownloadFWD} has expired in a servant peer before being completed (the servant has not forwarded all the requested file intervals subject of the delegated task to the requester node).
\item \textbf{DTNDownloadFWD completion:} a \textit{DTNDownloadFWD} has been completed in a servant peer (the servant has forwarded all the requested file intervals subject of the delegated task to the requester node).
\end{itemize}

Every event is saved in the following format:
\begin{center}
\textit{Sim\_time	event\_description	event\_details}
\end{center}
The following are some examples:

\begin{center}
\textbf{15538.0000	PendingDownload	A32 to E476}
\end{center}
Refers to the task delegation from the node A32 to the node E467, 15538 seconds after the beginning of the simulation.
\\

\begin{center}
\textbf{38559.5000 PendingDownload completed in F630 (317200191 requested by A32)}
\end{center}
Refers to the completion of the \textit{DTNPendingDownload} in the servant node F630 at time 38559.5. It was delegated by the node A32 and the id of the searched file was \textit{317200191}
\\

\begin{center}
\textbf{213738.0000	DownloadFWD expired in F523 (317200191 requested by A32)}
\end{center}
Refers to the expiration of the \textit{DownloadFWD} in the servant node F523 at time 213738. It was delegated by the node A32 and the id of the searched file was \textit{317200191}
\\

\paragraph{DataTransferLog}
\textit{DataTransferLog} module is another log module, which keeps track of events regarding data transfer between nodes in the simulation. These can occur during the execution of several activities:
\begin{itemize}
\item \textbf{VirtualFile:} when the requester peer is in range with a node carrying the searched file, and the data transfer take place. Multiple data transfer events can be generated, if there are several peers with the searched file in range.
\item \textbf{DTNPendingDownload:} when a servant peer downloads, from a node carrying the searched file, some requested intervals contained in the delegated task.
\item \textbf{DTNDownloadFWD:} when a servant peer forwards the result of a delegated \textit{DTNPendingDownload} to the client peer.
\end{itemize}
Data transfer events are saved in the following format:
\begin{center}
\textit{Sim\_time	from\_address	to\_address	data\_transferred (in bytes)}
\end{center}
e.g.
\begin{center}
\textbf{98829.0000	G714	A17	2750001}
\end{center}
Refers to the data transfer from node G714 to node A17, 98829 seconds after the start of the simulation. 2750001 bytes have been transferred in this event.
\\

\paragraph{FileGatheringReport}
\textit{FileGatheringReport} module is the most important report module used during our simulations, because it summarizes all key aspects of a single simulation. In detail, these are the values tracked by this module:
\begin{itemize}
\item \textbf{Total data:}	the total amount of data traffic exchanged between servant peers trying to satisfy a delegated task, file possessor peers and the data forwarding quantity toward the requester node.
\item \textbf{VirtualFile created:}	number of \textit{VirtualFile} task created during the simulation
\item \textbf{VirtualFile delegated:} how many times a task has been delegated to a servant peer
\item \textbf{PendingDownloads completed:} how many of the delegated tasks have been completed i.e. how many times the servant peer has been able to locate and download all the data intervals requested in the \textit{DTNPendingDownload} 
\item \textbf{PendingDownloads expired:} how many of the delegated tasks expired before the servant peer being able to find and download all the data intervals requested in the \textit{DTNPendingDownload} 
\item \textbf{DownloadFWDs expired:} how many \textit{DownloadFWDs} expired i.e. the servant peer downloaded all the data intervals requested in the \textit{DTNPendingDownload} but it was not able to forward them to the requester peer before the task expiration
\item \textbf{DownloadFWDs returned:} how many \textit{DownloadFWDs} has been forwarded correctly to the requester peer
\item \textbf{VirtualFile completed:} how many of the created \textit{VirtualFile} has been completed
\item \textbf{First VirtualFile satisfied:}	the time (in seconds) passed from the file request creation before the \textit{VirtualFile} has been completed
\item \textbf{Simulated time:} the simulated duration (in seconds) of the simulation
\item \textbf{Simulation time:}	the real duration of the simulation (in seconds)
\end{itemize}
Saving all these values for every simulation is useful in order to make later analysis such as averages, minimum and maximum values and to find the best or worst case for a large number of simulations.

\paragraph{M2MShareMapCoverageReport}
In order to evaluate the achieved explored area using different delegation strategies we implement the \textit{M2MShareMapCoverageReport}. This module divides the simulated map in squares with size of 10 x 10 meters and counts how many times each square has been visited during the simulation. When a node moves, the module saves its position by adding 1 to the value of the related square. The report is configurable, in order to save only movements related to nodes employing different delegation strategies:
\begin{itemize}
\item the node with the initial file request issued by the user
\item nodes acting as servants, i.e. nodes with delegated tasks received
\item all nodes in the simulation
\end{itemize}
The reason we decided to save all nodes movements, in some simulations, is to obtain a \textit{control set} to evaluate the maximum area that can be explored using certain parameters. By doing so for a high number of simulations we see which parts of the map can be explored (streets, buildings, parks) and which are not (like rivers, the sea or places not reachable with nodes movement).

%
%\paragraph{DelegationGraphvizReport}
%The last report module implemented for our simulation is a module responsible for the creation of reports describing the delegation history of the simulation using a Graphwiz graph. These graphs are described using the DOT language and can be displayed using the \textit{Graphwiz Graph Visualization Software}\footnote{http://www.graphviz.org/}. These graphs are useful to get a visual feedback about the delegations of tasks done and the status of that tasks, especially in case of multi-hop delegations. For every peer involved in delegations can be seen:
%\begin{itemize}
%\item if it receive the delegation (obviously)
%\item if it complete the delegated task
%\item if it forwarded back the result of the delegated task
%\end{itemize}
%In figure \ref{graph-example}
%\figuremacro{graph-example}{Graphviz graph example}{An example of graph generated using the output of DelegationGraphvizReport module}{}


\section{M2MShare Implementation}
In this section we describe the implementation of modules composing M2MShare into the ONE simulator. As said earlier in Section \ref{simulatore}, every routing protocol is an extension of a common superclass named \textit{MessageRouter} and, according to the the ONE philosophy, a new routing module can be inserted extending that class and it can be consequently used in configuration time to set the routing behaviour of nodes.
\\

M2MShare implementation consists in the main class extending \textit{MessageRouter}, named \textit{M2MShareRouter}, and several other classes contained in the package \textit{routing.m2mShare}.

\paragraph{M2MShareRouter}
As mentioned earlier, this is the main class of M2MShare implementation. Its first responsibility is to load the related settings as set in configuration files (see Section \ref{configurazioneONE}) and to initialize all the modules needed for the execution of the protocol. These entities will be described later and are responsible of the different aspects of the protocol. Values used to initialize the settings are read from configuration files and all are included in the settings namespace \textit{M2MShareRouter}. Every parameter has a default value, used in case nothing is not specified in any configuration file. Here the available parameters are shown and, shown in brackets, the relative default value:
\begin{itemize}
\item \textbf{M2MShareRouter.frequencyThreshold [2]} indicates the minimum number of encounters needed to elect a peer as servant and consequently delegate a task to it 
\item \textbf{M2MShareRouter.scanFrequency [10]} indicates how many seconds the \textit{PresenceCollector} must to wait between one scan and the next
\item \textbf{M2MShareRouter.delegationType [1]} indicates the type of delegation strategy used. It accepts three values:
\begin{itemize}
\item \textbf{0:} do not use delegation and file exchange is initiated only when a peer holding the requested data file is found in the reach area 
\item \textbf{1:} use the M2MShare technique where unaccomplished tasks are delegated only to peers which exceed the \textit{frequencyThreshold} value
\item \textbf{2:} use the trivial technique where unaccomplished tasks are delegated to each encountered peer
\end{itemize}
\item \textbf{M2MShareRouter.fileDivisionType [1]} indicates the type of file division strategy used. It accepts three values:
\begin{itemize}
\item \textbf{0:} for every file transfer is requested the entire file
\item \textbf{1:} use the M2MShare technique to choose the initial download point in the requested file
\item \textbf{2:} randomly choose the initial download point in the requested file
\end{itemize}
\item \textbf{M2MShareRouter.useBroadcastModule [true]} used to enable/disable the broadcast module. Could be useful to speed-up simulations, at the cost of a loss of precision
\item \textbf{M2MShareRouter.delegationDepth [1]} indicates the maximum number of delegation hops can be used for delegations, starting from the initial requester
\item \textbf{M2MShareRouter.stopOnFirstFileRequestSatisfied [false]} used to make the simulation stop when the first File Request has been satisfied. Can be useful in simulations in which we are not interested in what happens after the File Request is satisfied
\end{itemize}

%The class \textit{M2MShareRouter} also extends the superclass \textit{MessageRouter} and doing so it override the main method of this class: \textit{update()}. As said in Section \ref{esecuzioneONE}, that method is called one time for every simulated time interval, after the update of movement and connections of the node. In \textit{M2MShareRouter} the only action executed in \textit{update()} is to call the method \textit{runUpdate()} in module \textit{Scheduler}, described in section \ref{schedulerImplementazione}.

\paragraph{PresenceCollector}
The module \textit{PresenceCollector} is responsible for gathering information about in-reach area devices and to track encounters between other nodes to realize the election strategy selected for the current node in the simulation. Encounters data is gathered scanning for in-reach devices with a fixed frequency. When another node exceeds a value named \textit{Frequency Threshold}, it can be elected as a servant and receive unaccomplished tasks as delegations from the current node. Encounters data is saved in a structure called \textit{Servant List} which uses a replacement policy to manage peer slots in the list. If a peer has been encountered for a period greater than a value, called \textit{Probation Window}, without being elected as servant, its slot is freed to give other peers the opportunity to be elected as servants. \textit{PresenceCollector} also includes a tuning algorithm which adapts \textit{Frequency Threshold} and \textit{Probation Window} values at the beginning of every simulated day, according to what observed during the previous day.


\paragraph{Activities}
A single node in M2MShare system can execute several kinds of tasks. In our implementation, each of them is an implementation of interface \textit{DTNActivity} and characterizes one task type respect to another for what concerns their execution. 
\\

In a \textit{VirtualFile} task type, the execution is focused on searching in-reach devices for the file the user asks to find and download. A VirtualFile includes an IntervalMap used to store those file pieces that are still missing. When a VirtualFile task is completed, a new DTNFile is created and saved in the node's file system. 
\\

A VirtualFile can be delegated to a servant peer. When a servant node receives a delegated task, it creates and queues a \textit{DTNPendingDownload} activity. In its execution, the servant node searches in-reach devices for requested intervals of the file described in the delegated task. When a DTNPendingDownload task is completed, the servant node creates end schedules a \textit{DTNForward} task for execution.
\\

In this task's execution, the servant node looks for the client node that delegated it the VirtualFile task. When this node is found, the servant node notifies that it is ready to forward the output of the DTNPendingDownload task. The client can request all or only a subset of intervals the servant downloaded.
\\

Tasks active in servant nodes (DTNPendingDownload and DTNForward) have a TTL value, calculated using client's Probation Window value. When this value is exceeded, the task is deleted from queues and never again scheduled for execution.
\\

With our addition to M2MShare first version, a task can be delegated with more than one hop. To do so we allow M2MShareRouter to delegate also incomplete DTNPendingDownload activities. The result of this delegation, in servant peer, is the creation of a new DTNPendingDownload task. A delegation history list is included in every delegated task. When this second-level task is completed, a new DTNForward task is created in the servant peer. Its execution consists in returning the output of the second-level delegation to one of the nodes in delegation history list. This allows to return the output to the initial requester node faster than returning it following the inverse order of delegation history list.
\\

To avoid generating an high number of delegations, we set a trial period before a servant node delegate again a pending task. It waits one day and during this period it tries to complete the task by itself. If the task is still incomplete after one day, the node proceeds to delegate it for another hop. The maximum number of hops for each delegation is configurable for every simulation. To avoid delegation cycles, we implemented an anti-cycle system similar to the one used in AODV \cite{aodv} which prevents delegating a task to a node that is already acting as servant for it. 

\paragraph{File Division Strategies}
To evaluate the efficiency of M2MShare file division strategy, we implemented a \textit{IntervalMap} according to the one described in Section \ref{descrFileDivisionStrategy}. This structure includes not yet downloaded intervals for a file and is used by our system's Activities to implement the adopted file division strategy. Its behaviour can be set through configuration files for every simulation. This allow emulations of different file division strategies and comparing their efficiency in several simulations:
\begin{itemize}
\item \textbf{M2MShare:} the strategy with dynamic starting point, described in Section \ref{descrFileDivisionStrategy}
\item \textbf{iM:} where each file transfer starts from the beginning of the file
\item \textbf{rM:} where each file transfer starts from a random point within the file
\end{itemize}


\paragraph{Data Transfer}
To emulate data transfers between nodes in the simulated network, we implement a entity named \textit{Communicator}. This entity is used by Activities to emulate the downloading of pieces of files from a file possessor, or the forwarding of a pending task's result to a client node. There is no 1-1 relation between Communicators and Activities. A task can use more than one Communicator when try to simultaneously download pieces of the same file from different sources within communication range. Communicator behaviour is related to the adopted file division strategy, as it influences the starting point of file transfers. Using Communicators allows us to emulate data transfers only when a connection between two nodes is available and to notify information about transfers to related report modules.

%\paragraph{Scheduler}
%\label{schedulerImplementazione}
%\paragraph{Executor}
%\paragraph{Communicator}
%
%\paragraph{BroadcastModule}
 
\section{Analysis}
To evaluate M2Mshare efficiency we need to analyse its performance compared with other strategies or with different versions of M2MShare, as in multi-hop simulations. To do so we repeat each simulation several times changing the random module generators seeds. We do so to achieve results independent from nodes movement and the starting point. For each simulation we save some report files, generated using report modules described in Section \ref{descrReports}. Several analyses have been made over these reports using a set of classes we implemented for this purpose. 
\\

Repeating each simulation with different strategies we are able to compare their performance related to several values. Using our report and analysis modules, for each single simulation we are able to evaluate:
\begin{itemize}
\item number of File Requests created
\item how many tasks have been delegated during the simulation
\item how many delegated tasks have been completed
\item how many delegated tasks expired before completion
\item how many delegated tasks have been completed but expired before the result was forwarded to the requester node
\item number of File Request satisfied
\item after how much time a File Request has been satisfied
\item the length of simulated time
\item the length of the simulation (in real time)
\end{itemize} 

For each set of simulations repeated changing random generators seeds, we are able to evaluate 
\begin{itemize}
\item minimum value
\item maximum value
\item average value
\end{itemize} 
related to each of the parameters above.
 
% ---------------------------------------------------------------------------
%: ----------------------- end of thesis sub-document ------------------------
% ---------------------------------------------------------------------------


% this file is called up by thesis.tex
% content in this file will be fed into the main document

%: ----------------------- name of chapter  -------------------------
\chapter{Simulations and results}\label{simulazione} % top level followed by section, subsection


%: ----------------------- paths to graphics ------------------------

% change according to folder and file names
%\graphicspath{{2-Consorzi/images/}}


%: ----------------------- contents from here ------------------------
In previous chapters we focused on the concepts, related technologies and implementation realized of the protocol, M2MShare, subject of the study of this thesis. In this section we'll describe the simulations executed and analysis with relative results. For every type of analysis we'll describe the related simulations, the configuration of the simulated world and the other protocols compared to our M2MShare. Finally, for every analysis, we'll show the results in a graphical way with a textual interpretation.
\\

We repeat each of the simulation scenarios several times in order to achieve more accurate results, independent of the initial positioning of the requester peer in search for that particular data file and independent of the initial positioning of the searched file copies. Each scenario is run using a different random seeds to initialize the movement model and for every seed the simulation is repeated using the three compared protocols.

Before proceeding further, we introduce the reader to the adopted terminology and some "must know" configuration parameters:
\begin{itemize}
\item \textit{Population}: refers to the number of nodes in the simulation which emulate people operating M2MShare. This number don't include nodes which emulate public transport, like buses or trams. People are distributed in districts of the map, described in Section \ref{mappaONE}.

\item \textit{File size}: refers to the size of the data file the user is looking for in the simulation 

\item \textit{File popularity}: refers to how many copies of the data file the user is looking for in the simulation are present at the beginning of the simulation. Nodes initially carrying the file are randomly chosen, using the DTNFileGenerator described in Section \ref{fileGeneratorImplementazione} and can be uniformly chosen between all the active nodes or in a subset of them.

\item \textit{Delegation Type}: refers to the type of delegation used in the group of simulations. It can be of three types:
\begin{itemize}
\item \textbf{No\_delegation:} do not use delegation and file exchange is initiated only when a peer holding the requested data file is found in reach area 
\item \textbf{M2MShare:} use the M2MShare technique where missing tasks are delegated only to peers which exceed the \textit{frequencyThreshold} value
\item \textbf{Delegation\_to\_all:} use the trivial technique where missing tasks are delegated to each encountered peer
\end{itemize}

\item \textit{Delegation Depth}: how many times a delegated task can be re-delegate to other peers.

\item \textit{File Division Strategy}: refers to the type of file division strategy used during file transfer between nodes. It can be of three types:
\begin{itemize}
\item \textbf{M2MShare:} use the M2MShare technique to choose the initial download point in the requested file
\item \textbf{iM:} for every file transfer is requested the entire file
\item \textbf{rM:} randomly choose the initial download point in the requested file
\end{itemize}

\item \textit{Nr. of simulations}: how many times the simulation has been repeated with different movement and file generation seeds.

\item \textit{Simulated time}: the simulated length of simulations

\end{itemize}

\newpage
\section{Delegation efficiency}
\label{analisiDelegationEfficiency}
In this analysis we evaluate the efficiency in using delegation versus not using it. Simulations configuration are shown in the following table:
\begin{table}[h]
\begin{center}
\begin{tabular}{|l|r|}
\hline
\bfseries Population & 1000 \\
\hline
\bfseries File size & 3.0 MB \\
\hline
\bfseries File popularity & 50 copies uniformly chosen \\
\hline
\bfseries Delegation type & No\_delegation, M2MShare, Delegation\_to\_all \\
\hline
\bfseries Delegation depth & 1 \\
\hline
\bfseries File Division Strategy & M2MShare \\
\hline
\bfseries Nr. of simulations & 40 x 3\\
\hline
\bfseries Simulated time & One week \\
\hline
\end{tabular}
\end{center}
\end{table}

We compare the efficiency of our system (M2MShare) employing delegations against two other systems using different strategies:
\begin{itemize}
\item \textit{No\_delegation:} system which does not employ delegations and file exchange is initiated only when a peer holding the requested data file is found in reach area of file requester.
\item \textit{Delegation\_to\_all:} system employing delegations but instead employs the trivial technique where missing tasks are delegated to each encountered peer.
\end{itemize}

The metric we study is the found time (Ft) for a generic data file which is the time interval between the first delegation made and the time an output return for that specific file is
received. If no delegations are made and the first file request is satisfied by a direct file possessor the Ft is equal to zero. For each of the above scenarios we also measure the:
\begin{itemize}
\item number of delegations used: representing the number of tasks the requester peer has delegated for that particular data download;
\item percentage of completed task: representing the number of delegated tasks completed (output returned) over all delegated tasks;
\item total data transferred: referring to the quantity of data traffic exchanged between servant peers trying to satisfy a delegated task, file possessor peers and the data forwarding quantity toward the requester node.
\end{itemize}

\begin{figure}[ht]
\begin{minipage}[b]{0.5\linewidth}
\centering
\includegraphics[scale=0.5]{grafici/tempi.eps}
\caption{Average, min. max found time employed by each strategy in finding the required data file.}
\label{graficoTempiVF}
\end{minipage}
\hspace{0.5cm}
\begin{minipage}[b]{0.5\linewidth}
\centering
\includegraphics[scale=0.5]{grafici/delegheFatte.eps}
\caption{Average, min, max number of delegations employed by each delegation strategy.}
\label{graficoNumeroDeleghe}
\end{minipage}
\end{figure}


In Fig.\ref{graficoTempiVF} it is possible to see the advantage, in terms of found time, in using the delegation technique instead of not using it. The two systems employing delegations find the required file in less time in each simulation run at the expense of higher overhead in terms of bandwidth due to delegations. The system employing delegations to all encountered peers gets a better result on average, but at a cost of a higher number of delegated tasks (Fig.\ref{graficoNumeroDeleghe}). A higher number of delegated tasks imply more bandwidth used for searching the data file and potentially retrieving (if found) and forwarding it toward the requester.
\\

Fig.\ref{graficoNumeroDeleghe} makes a comparison between the two systems employing delegations by showing the number of overall employed task delegations till file download or simulation time expires. It is easy to see that M2MShare uses fewer delegations while achieving a higher percentage of completed delegated tasks (Fig. \ref{graficoPercDelegheRitornate}). This outcome is due to a conservative delegation strategy employed by M2MShare in delegating unsatisfied, unaccomplished tasks only to frequently encountered peers
(servants). Since we do not have any means to evaluate the ability of one servant to satisfy a file request what we do is delegate to encountered peers whom is expected to be encountered again in the future. The Delegation\_to\_all strategy contributes to higher overhead also due to completed tasks, ready to be returned toward the requester that unfortunately expire and are discarded before having the chance of encountering the data file requester.
\\

\begin{figure}[ht]
\begin{minipage}[b]{0.5\linewidth}
\centering
\includegraphics[scale=0.5]{grafici/percDeleghe.eps}
\caption{Percentage of completed previously delegated tasks against the number of overall delegations employed.}
\label{graficoPercDelegheRitornate}
\end{minipage}
\hspace{0.5cm}
\begin{minipage}[b]{0.5\linewidth}
\centering
\includegraphics[scale=0.5]{grafici/data.eps}
\caption{Average, min, max transferred data amount employed in each delegations technique.}
\label{graficoDataDiverseDel}
\end{minipage}
\end{figure}


In Fig.\ref{graficoDataDiverseDel} is shown the comparison between M2MShare and Delegation\_to\_all strategies in terms of transferred data quantities till file download or simulation time ends. It is straightforward to notice the higher overhead in terms of data transmissions introduced by delegating to all encountered peers whether M2MShare reduces the exchanged data quantity while still achieving the goal of acquiring the requested file. From the above results it is obvious that the delegation strategy serves its purpose by extending a peer reach area to other mobile disconnected networks where data content might be available therefore reducing the found time of a desired content.
Although this strategy introduces an overhead in terms of bandwidth usage, computation and power consumption we control these side effects by delegating only to frequently encountered peers whom are expected to be encountered again in the future.



\newpage 
\section{Delegation efficiency with variable file popularity}
In previous analysis we show the advantage, in terms of found time of the searched file, using M2MShare delegation strategy against avoid using delegation or delegate unaccomplished tasks to every meet node. The confrontation was made using a constant number of copies of the searched multimedia file uniformly distributed between all nodes in the simulation, i.e. 50 copies. The current analysis want to show the difference in performance using the three delegation strategies changing the initial file popularity of the searched file. To this aim, in these simulations we change the File popularity (Fp) of the multimedia file keeping constant the number of total nodes (N).
These are the settings used for the simulations:
\begin{table}[h]
\begin{center}
\begin{tabular}{|l|r|}
\hline
\bfseries Population & 1000 \\
\hline
\bfseries File size & 3.0 MB \\
\hline
\bfseries File popularity & 50, 100, 150, 200, 250, 300, 350, 400 copies \\
\hline
\bfseries Delegation type & No\_delegation, M2MShare, Delegation\_to\_all \\
\hline
\bfseries Delegation depth & 1 \\
\hline
\bfseries File Division Strategy & M2MShare \\
\hline
\bfseries Nr. of simulations & 40 x 8 x 3\\
\hline
\bfseries Simulated time & 48 hours \\
\hline
\end{tabular}
\end{center}
\end{table}
We change the File Popularity (Fp) value, from 5\% (50 copies) to 40\% (400 copies). When the Fp is low (with Fp £ 5\%) the system which don't use delegation is not able to find any piece of the file during the simulation time. We have indicated this in the chart by assigning to Ftavg a value of 48 h. With higher values of Fp, the first protocol is able to find the file thanks to the higher popularity of the requested file, but it takes more time than M2MShare and Delegation\_to\_all. Finally, with the highest values of Fp the performances of the three compared systems became very similar. 
\begin{figure}[ht]
  \begin{center}
    \includegraphics[scale=0.5]{grafici/percDeleghe.eps}
    \caption{Percentage of completed previously delegated tasks against the number of overall delegations employed.}
    \label{graficoPopVariabile}
  \end{center}
\end{figure}


\newpage
\section{Delegation efficiency with variable nodes population}
In previous analysis we considered as constant the number of nodes emulating people using M2MShare on their devices. In the current analysis we changed the total population of the simulations, observing how this affects the performance of compared systems.
\begin{table}[h]
\begin{center}
\begin{tabular}{|l|r|}
\hline
\bfseries Population & 100, 200, 400, 600, 800, 1000 \\
\hline
\bfseries File size & 3.0 MB \\
\hline
\bfseries File popularity & 5\%, 10\%, 50\% \\
\hline
\bfseries Delegation type & No\_delegation, M2MShare, Delegation\_to\_all \\
\hline
\bfseries Delegation depth & 1 \\
\hline
\bfseries File Division Strategy & M2MShare \\
\hline
\bfseries Nr. of simulations & 40 x 6 x 3 x 3\\
\hline
\bfseries Simulated time & 48 hours \\
\hline
\end{tabular}
\end{center}
\end{table}

In the first scenario (Fig.\ref{graficiTempiVF_Fp5}) we consider Fp = 5\%. The protocol not employing delegations (black line in the chart) is not able to find any piece of the file during the simulation time when the considered nodes in the scenario are equal or less than 400. We have indicated this in the chart by assigning to Ftavg a value of 48 h. This is due to the trivial strategy employed by the protocol and to the sparse environment. Even when able to find some node possessing the file (with $N \geq 600$), the time needed results bigger than using a strategy implementing delegations. Clearly, when increasing the file population (Fp), even the number of nodes in the population that posses the data file increases; as a result, the time to retrieve the file decreases for all solutions. A similar result is achieved also when considering a wider popularity for the required multimedia file (Fp = 10\%, in Fig.\ref{graficiTempiVF_Fp10}). However, in this case, the higher popularity of the requested file helps both solutions in finding the file possessor with a smaller Ftavg than in the previous scenario.  Finally, in Fig.\ref{graficiTempiVF_Fp50}, the performances of the compared solutions are very similar. This is due to the high file popularity among nodes (Fp = 50\%): the chances of eventually finding a file possessor in a short time are clearly much higher.

\begin{figure}[ht]
\begin{minipage}[b]{0.5\linewidth}
\centering
\includegraphics[scale=0.5]{grafici/percDeleghe.eps}
\caption{Average found time with $Fp = 5\%$}
\label{graficiTempiVF_Fp5}
\end{minipage}
\hspace{0.5cm}
\begin{minipage}[b]{0.5\linewidth}
\centering
\includegraphics[scale=0.5]{grafici/data.eps}
\caption{Average found time with $Fp = 10\%$}
\label{graficiTempiVF_Fp10}
\end{minipage}
\hspace{0.5cm}
\begin{center}
\begin{minipage}[b]{0.5\linewidth}
\centering
\includegraphics[scale=0.5]{grafici/data.eps}
\caption{Average found time with $Fp = 50\%$}
\label{graficiTempiVF_Fp50}
\end{minipage}
\end{center}
\end{figure}


\newpage
\section{Data redundancy}
In analysis in Section \ref{analisiDelegationEfficiency}, especially in Fig. \ref{graficoDataDiverseDel}, we show that our system is the most efficient with respect to data transmissions. Although using delegations introduces an overhead in terms of bandwidth usage we control this side effects by delegating only to frequently encountered peers whom are expected to be encountered again in the future. Another side effect caused by delegating tasks is the increasing of data redundancy in the whole network. For \textit{redundancy}, in this case, we mean storage space used in nodes involved in delegation system. \\ These are the settings used in simulations related to the study of this metric:

\begin{table}[h]
\begin{center}
\begin{tabular}{|l|r|}
\hline
\bfseries Population & 1000 \\
\hline
\bfseries File size & 3.0 MB \\
\hline
\bfseries File popularity & 50 copies uniformly chosen \\
\hline
\bfseries Delegation type & M2MShare, Delegation\_to\_all \\
\hline
\bfseries Delegation depth & 1 \\
\hline
\bfseries File Division Strategy & M2MShare \\
\hline
\bfseries Nr. of simulations & 40 x 2\\
\hline
\bfseries Simulated time & One week \\
\hline
\end{tabular}
\end{center}
\end{table}

Of course redundancy in a network composed only by nodes which don't use delegation is always zero. For this reason we compared for this study only the two systems which use task delegations. In Fig.\ref{graficoRidondanzaData} we show how change the average data redundancy during the progress of simulations. It is straightforward to notice the higher value introduced by delegating to all encountered peers whether M2MShare reduces the data redundancy quantity while still achieving the goal of acquiring the requested file. This is due to the number of contemporary active delegated tasks, shown in Fig.\ref{graficoDelegheAttive}, which is higher in the system which delegates task to all encounter nodes.

\begin{figure}[ht]
\begin{minipage}[b]{0.5\linewidth}
\centering
\includegraphics[scale=0.5]{grafici/percDeleghe.eps}
\caption{Average found time with $Fp = 5\%$}
\label{graficoRidondanzaData}
\end{minipage}
\hspace{0.5cm}
\begin{minipage}[b]{0.5\linewidth}
\centering
\includegraphics[scale=0.5]{grafici/data.eps}
\caption{Average found time with $Fp = 10\%$}
\label{graficoDelegheAttive}
\end{minipage}
\end{figure}

\newpage 
\section{File Division Strategy efficency}
In previous analysis we evaluate the efficiency of M2MShare delegation technique, but in the protocol the delegation system is not the only new innovation described. M2MShare provides a new file division strategy, described in Section \ref{descrFileDivisionStrategy}, where a file can be downloaded in pieces and a piece size varies. That file division strategy might add redundancy during data transfer as it can happen that overlapping data intervals are simultaneously downloaded by different servants. However, the fact that each servant is asked to download the file starting from different points allows reconstructing the whole file even if both downloaded just part of it. \\
In this analysis we compare our file division strategy with two other division strategies:
\begin{itemize}
\item iM: a strategy which requests at each file server the
entire file, always starting from its first byte;
\item rM: a strategy that randomly chooses the initial
download point in the file request.
\end{itemize}
These are the settings of the related simulations:
\begin{table}[h]
\begin{center}
\begin{tabular}{|l|r|}
\hline
\bfseries Population & 1000 \\
\hline
\bfseries File size & 3.0 MB, 10.0 MB, 25.0 MB \\
\hline
\bfseries File popularity & 50 copies uniformly chosen \\
\hline
\bfseries Delegation type & M2MShare \\
\hline
\bfseries Delegation depth & 1 \\
\hline
\bfseries File Division Strategy & M2MShare, iM , rM \\
\hline
\bfseries Nr. of simulations & 40 x 3 x 3\\
\hline
\bfseries Simulated time & One week \\
\hline
\end{tabular}
\end{center}
\end{table}

We considered the average, min and max transferred data amount employed in each file division strategy. We repeated the simulations 40 times, changing random seeds, for every file division strategy in order to achieve more accurate results and we did this using three different sizes of the searched file. 
\\
In earlier studies some test were done to evaluate the efficiency of M2MShare file division strategy, but task delegation was not considered because simulated scenarios did not take into consideration disconnections due to mobility or interferences that happen in real world
communication. That has only been possible with the current implementation of the protocol, in a network simulation environment which can simulate movement of nodes emulating people using M2MShare.
As we can see from the graphics below our division strategy has the least redundancy during data transfer, especially considering big-sized files.

\begin{figure}[htpb]
  \begin{center}
    \includegraphics[scale=1]{grafici/dataDFS_3MB.eps}
    \caption{Average, min, max transferred data amount using different file division strategies and 3.0 MB file size.}
    \label{graficoDataFDS_3MB}
  \end{center}
\end{figure}

\begin{figure}[htpb]
  \begin{center}
    \includegraphics[scale=1]{grafici/dataDFS_10MB.eps}
    \caption{Average, min, max transferred data amount using different file division strategies and 10.0 MB file size.}
    \label{graficoDataFDS_10MB}
  \end{center}
\end{figure}

\begin{figure}[htpb]
  \begin{center}
    \includegraphics[scale=1]{grafici/dataDFS_25MB.eps}
    \caption{Average, min, max transferred data amount using different file division strategies and 25.0 MB file size.}
    \label{graficoDataFDS_25MB}
  \end{center}
\end{figure}

%\label{analisiLabel}
 
% ---------------------------------------------------------------------------
%: ----------------------- end of thesis sub-document ------------------------
% ---------------------------------------------------------------------------


% this file is called up by thesis.tex
% content in this file will be fed into the main document

%: ----------------------- name of chapter  -------------------------
\chapter{Conclusioni}\label{conclusioni} % top level followed by section, subsection


%: ----------------------- paths to graphics ------------------------

% change according to folder and file names
%\graphicspath{{2-Consorzi/images/}}


%: ----------------------- contents from here ------------------------
LE conclusioni di tutto
 
% ---------------------------------------------------------------------------
%: ----------------------- end of thesis sub-document ------------------------
% ---------------------------------------------------------------------------




\cleardoublepage 	 %va a pagina nuova iniziando dalla pagina di destra

%%%%%%%%%%%%%%%%%%%%%%%%%%%%%%%%%%%%%%%%%%%%%%%%%%%%%%%

%%%%%%%%%Facoltativi%%%%%%%%%%%%%%%%%%%%%%%%%%%%%%%%

%%%%%%%%%%%%%%%%%%CONCLUSIONI%%%%%%%%%%%%%%%%%%%%%%%%%%

%\rhead[\fancyplain{}{\bfseries                                              %imposta l'intestazione di pagina destra
%CONCLUSIONI}]{\fancyplain{}{\bfseries\thepage}}
%\lhead[\fancyplain{}{\bfseries\thepage}]{\fancyplain{}{\bfseries            %imposta l'intestazione di pagina sinistra
%CONCLUSIONI}}
%\addcontentsline{toc}{chapter}{Conclusioni}                                 %aggiunge la voce Conclusioni nell'indice
%\input{Conclusioni}                                                         %preleva il file in cui ci sono le conclusioni
%\cleardoublepage

%%%%%%%%%%%%%%%%%%%%%%%%%%%%%%%%%%%%%%%%%%%%%%%%%%%%%%%


%%%%%%%%%%%%%%%%%%%%APPENDICE%%%%%%%%%%%%%%%%%%%%%%%%%%%%%5

%\rhead[\fancyplain{}{\bfseries                                              %imposta l'intestazione di pagina destra
%APPENDICE}]{\fancyplain{}{\bfseries\thepage}}                               %imposta l'intestazione di pagina
%\lhead[\fancyplain{}{\bfseries\thepage}]{\fancyplain{}{\bfseries            %imposta l'intestazione di pagina (si pu� cambiare la scritta appendice con il suo nome in modo da far apparire nella intestazione della pagina sinistra il nome dell'appendice)
%APPENDICE}} \addcontentsline{toc}{chapter}{Appendice: Data Sheet}
%\input{Appendice}
%\cleardoublepage

%%%%%%%%%%%%%%%%%%%%%%%%%%%%%%%%%%%%%%%%%%%%%%%%%%%%%%%


%%%%%%%%%%%%%BIBLIOGRAFIA%%%%%%%%%%%%%%%%%%%%%%%%%%%%%%

%\begin{thebibliography}{90}                                                 %crea l'ambiente bibliografia
%\rhead[\fancyplain{}{\bfseries                                              %imposta l'intestazione di pagina
%BIBLIOGRAFIA}]{\fancyplain{}{\bfseries\thepage}}                            %imposta l'intestazione di pagina
%\lhead[\fancyplain{}{\bfseries\thepage}]{\fancyplain{}{\bfseries            %imposta l'intestazione di pagina
%BIBLIOGRAFIA}}
%\addcontentsline{toc}{chapter}{Bibliografia}                                %aggiunge la voce Bibliografia nell'indice
%\input{Bibliografia}
%\end{thebibliography}
%\cleardoublepage

\addcontentsline{toc}{chapter}{References}
\nocite{*}
\bibliographystyle{unsrt}
\bibliography{bibliografia}


%%%%%%%%%%%%%%%%%%%%%%%%%%%%%%%%%%%%%%%%%%%%%%


\rhead[\fancyplain{}{\bfseries\leftmark}]{\fancyplain{}{\bfseries\thepage}} %imposta l'intestazione di pagina
\lhead[\fancyplain{}{\bfseries\thepage}]{\fancyplain{}{\bfseries 
FIGURE INDEX}}
%
\listoffigures                                                              %crea l'elenco delle figure
\addcontentsline{toc}{chapter}{Figures Index}                         %aggiungilo all'indice
\cleardoublepage


\rhead[\fancyplain{}{\bfseries\leftmark}]{\fancyplain{}{\bfseries\thepage}} %imposta l'intestazione di pagina
\lhead[\fancyplain{}{\bfseries\thepage}]{\fancyplain{}{\bfseries
TABLES INDEX}}
\listoftables                                                               %crea l'elenco delle tabelle
\addcontentsline{toc}{chapter}{Tables Index}                        %aggiungilo all'indice
\cleardoublepage                                                            %non numera l'ultima pagina sinistra

%%%%%%%%%%%%%%RINGRAZIAMENTI%%%%%%%%%%%%%%%%%%%%

%\rhead[\fancyplain{}{\bfseries                                              %imposta l'intestazione di pagina
%RINGRAZIAMENTI}]{\fancyplain{}{\bfseries\thepage}}                          %imposta l'intestazione di pagina
%\lhead[\fancyplain{}{\bfseries\thepage}]{\fancyplain{}{\bfseries            %imposta l'intestazione di pagina
%RINGRAZIAMENTI}}
%\chapter*{Ringraziamenti}
%\thispagestyle{empty}
%\input{Ringraziamenti}
%\addcontentsline{toc}{chapter}{Ringraziamenti}                              %aggiunge la voce Ringraziamenti nell'indice
%\cleardoublepage

%%%%%%%%%%%%%%%%%%%%%%%%%%%%%%%%%%%%%%%%%%%%%%


\end{document}
