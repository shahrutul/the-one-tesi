% this file is called up by thesis.tex
% content in this file will be fed into the main document

%: ----------------------- name of chapter  -------------------------
\chapter{Modelli di movimento}\label{movimento} % top level followed by section, subsection


%: ----------------------- paths to graphics ------------------------

% change according to folder and file names
%\graphicspath{{2-Consorzi/images/}}


%: ----------------------- contents from here ------------------------
One important characteristic to achieve a good level of realism in DTN simulations is to simulate movement of nodes inside the simulated world. In Delay Tolerant Networks, movement of nodes is essential for the performance of the network, since end-to-end connectivity does not always exist and packets are delivered in a store and forward manner. For this reason is fundamental to simulate, with a good level of realism, movements behaviour of nodes inside the simulation.
\\

Capturing movement accurately in the real usage scenarios is also needed for a reliable assessment of a new protocol. Movement can be captured in simulations using real movement traces or synthetic traces generated by a movement model.

\section{Real user traces VS Synthetic models}
The most realistic user movement or contact patterns are the ones that happen in the real world. Therefore, several studies have focused on tracking user movement to be able to use the traces later directly for simulations or to learn more about which characteristics are common in real user behaviour.
Real movement traces have usually been obtained by analysing WLAN access point data or
having users carry around devices equipped with GPS modules. Contact traces have
mostly been collected from real world experiments where users have been carrying
around Bluetooth devices tracking other devices within range.

There are some problems about directly using the collected traces in simulations. Real world traces are usually from very specific environments like university campuses areas. These traces are not so generic to be used to simulate movement in a different environment, like a centre of a city, because behaviour and movement patterns of interested people are different. \\

Real trace are generally been obtained by analysing WLAN access point data or
having users carry around devices equipped with GPS modules. In the first case only certain areas have WLAN coverage and so movements outside access points communication range are not recorded. Using GPS modules limits mobility recording only to outdoor movements. 


When the locations of users are derived from access point data, the result is roughly building level granularity [16]. This is because users are not always connected to the closest access point and the movement between access points is difficult to capture. Most of the devices connecting to WLAN access points in these experiments are laptops and PDAs which are not always carried by the users and are not necessarily always turned on. Whether this characteristic is wanted in the simulations depends on what is modeled and simulated. Paper \cite{ImpactofHumanMobility} argues that the on/off times are an important characteristic of wireless users that needs to be taken into account and modeled for simulations. 

It is also practical to have a model with configurable parameters to work with. Sometimes a protocol or an application developer wants to test how the protocol or application performs when certain parameters change, to better be able to find out weaknesses and strengths. Sometimes researchers also need a very simple model to work with, to be able to use it in mathematical proofs for various theorems.


Synthetic models are often preferred since they are easier to work with than real user traces. Moreover, real user traces are rarely available for the environment to be modelled. Additionally, researchers want to do sensitivity analysis to find out how protocols and applications perform under different conditions. This is not possible with real user traces unless a parameterized model has been successfully extracted. 
There are two types of synthetic movement models that have been proposed for these
analyses — generic high level models that aim to produce movement accurate enough
with statistical measures, and models that describe incidental scenarios, hoping for a
more accurate depiction of single devices.
While efficient to use in simulations, the high level models, such as Random Waypoint
[18], often imply that the scenarios for which the protocols are simulated have huge
numbers of nodes, so that the relevant protocol features are given statistically realistic
distributions of events. For scenarios with few nodes, the differences between different
usage scenarios become more significant. Thus, movement models that depict more
precisely some specific types of movement are needed.
Traditionally, the approach to create a movement model has been to identify a certain
characteristic of mobility and to create a mathematical model describing the movement
at a high level. Such characteristics can be speed distribution, social relationships
between nodes or favorite locations nodes will visit. These types of movement have
very few details and the movement is homogeneous in the sense that every node is
moving according to the same rule.

Before we go any further, we introduce two new terms: locality and heterogeneity. By
locality we mean the tendency of nodes spending most of their time within a small area.
Thus, a movement model where each node's movement is restricted to a small area has
high locality. By heterogeneity, we mean different movement patterns and properties
between nodes. Measuring heterogeneity in a specific context is not straightforward,
but in our experiments we will later on talk about differences in contact patterns in
terms of how often a node encounters new nodes compared to the fraction of earlier
encountered nodes.

Hsu and Helmy [26] show by studying real user traces that nodes are very often turned
on/off and only visit a small portion of the WLAN access points in campus areas.
Moreover, they find that node mobility while using network is very low and one node
only meets a small portion of all other nodes in the area. These types of characteristics
are usually not captured in movement models. Furthermore, they reveal repetitive
patterns with a period of one day and heterogeneity among nodes. Although
heterogeneity and repetitiveness has been modeled, most simple movement models do
not. According to Hsu and Helmy, the biggest issue with most synthetic models is that
they are not capturing such characteristics as heterogeneous behavior, switching
devices on/off or relationships between users.
We noticed that even though most of the known features of movement have been
modeled, no model exist where all or many of the features have been combined. Our
approach is to combine these different elements to create a new movement model.



\section{Random Map-Based Movement}
The simplest map-based mobility model is Random Map-Based Movement (MBM). Nodes adopting this model moves randomly but always in streets described in the map. This is done walking from one map node, i.e. an address on a street or a crossroad, to the other by always randomly selecting one of the directly connected map nodes. Result of this behaviour is a random movement in which there are more contacts between nodes than in Random Walk model. This is due to the restriction to node movement to follow streets described in the map.


\section{Shortest Path Map-Based Movement}
Shortest Path Map-Based Movement (SPMBM) is another mobility model that uses a map-described environment to restrict node movement. With this model, nodes chooses their destination randomly inside the map, then calculate the shortest path to reach it and finally walk along this path. Dijkstra's algorithm is used to calculate the path. Destinations can be chosen in a totally random way or from a set of Points of Interests (POI). This can be useful to emulate interesting places like restaurants, monuments or shops. This model adds one level of realism, compared to MBM, 

\section{Routed Map-Based Movement}
Un modello che invece della casualità nel movimento utilizza dei percorsi predeterminati è il Routed Map-Based Movement (RMBM). I nodi che adottano questo modello si muovono lungo rotte predefinite, per tutta la durata della simulazione, rendendo RMBM utile per rappresentare degli spostamenti ripetitivi, come ad esempio quelli di autobus, tram o treni.

\section{Working Day Movement Model}
\label{descrWDM}
I modelli finora descritti sono senz'altro di semplice comprensione e realizzazione, oltre ad essere molto efficienti per quanto riguarda le prestazioni, ma non forniscono una realistica rappresentazione del movimento umano, soprattutto per quanto riguarda i valori di inter-contact e contact time.
\\
Un modello che genera dei valori più realistici per questi parametri, rappresentando più realisticamente quindi il movimento umano, è il Working Day Movement Model (WDM). Come il nome può fare intuire, questo modello simula gli spostamenti compiuti da una persona durante una tipica giornata lavorativa e in \cite{articoloWdm}, dove è descritto il modello, è evidenziato anche come i valori generati seguano realisticamente quelli trovati utilizzando dati di spostamento provenienti da tracce reali. 
\\
Una giornata simulata comprende le seguenti attività principali svolte dai vari nodi:
\begin{itemize}
\item dormire a casa
\item lavorare in ufficio
\item uscire alla sera con gli amici
\end{itemize}

Ovviamente le attività potrebbero cambiare enormemente a seconda dello stile di vita e del lavoro svolto dalle singole persone, ma queste tre attività sono le più comuni e possono essere associate alla tipica giornata di una gran quantità di persone.
La ripetitività giornaliera delle azioni e il fatto di svolgerle in luoghi comuni a più persone permette la formazione spontanea di comunità: persone che vivono e dormono nella stessa casa formeranno una famiglia, più persone che lavorano nello stesso ufficio agli stessi orari saranno colleghi di lavoro, mentre degli amici si possono trovare ad orari comuni alla sera per uscire assieme.
La creazione di queste comunità non viene mostrata da modelli più semplici quali RMBM o SPMBM.
\\

Per la simulazione delle attività giornaliere, WDM utilizza dei sottomodelli dedicati, oltre a dei sottomodelli preposti a simulare gli spostamenti fra un'attività e l'altra. Una persona si potrà quindi spostare a piedi, in auto o utilizzando i mezzi pubblici, a seconda della propria disponibilità e convenienza. Il fatto di muoversi da soli o in gruppo (prendendo lo stesso bus o camminando assieme la sera) permette di avere dei comportamenti eterogenei e quindi migliorare ulteriormente la realisticità degli spostamenti compiuti dai vari nodi.
\\

\subsection{Esempio di giornata}
Durante una tipica giornata il punto di partenza per ogni nodo è la propria abitazione. Ogni nodo ha un orario di sveglia assegnato, generato utilizzando una distribuzione normale con media pari a 0 e deviazione standard impostabile in fase di configurazione, che indica l'orario in cui la persona uscirà di casa. Il valore viene generato per ogni ad inizio simulazione, rimarrà lo stesso per tutti i giorni successivi e la differenza fra i valori di vari nodi sta ad indicare la differenza fra i diversi stili di vita nella vita reale (ad esempio una persona che impiega pochi minuti a prepararsi la mattina rispetto a chi impiega ore anche solo per fare colazione).
\\ 

Una volta usciti di casa, i vari nodi si dirigono al lavoro utilizzando l'auto (se disponibile) o a piedi oppure utilizzando i mezzi pubblici, a seconda di quale sia il metodo più conveniente. Conseguentemente alla scelta del mezzo di trasporto viene utilizzato il corrispondente sottomodello.
\\

Una volta raggiunto il luogo di lavoro, la persona ci resta per la durata della sua giornata lavorativa e quindi decide, con una determinata probabilità, se tornare direttamente a casa o spostarsi per un'attività serale. Anche in questo caso gli spostamenti vengono gestiti utilizzando i corrispondenti sottomodelli.


\subsection{Home Activity Submodel}
Ogni nodo ha una posizione impostata come Home Location, che viene utilizzata come punto di partenza alla mattina e punto di ritorno alla sera: una volta tornata a casa una persona si muove per una breve distanza e poi resta ferma fino all'orario di risveglio, la mattina successiva. Questo comportamento non è un errore, ma simula il fatto di lasciare il telefono su di un tavolo o in carica fino al momento di uscire nuovamente di casa, mentre la persona svolge le normali attività domestiche come mangiare, guardare la TV o dormire.

\subsection{Office Activity Submodel}
Il sottomodello relativo all'attività lavorativa è un modello bidimensionale che simula il comportamento di una persona all'interno di un ufficio, in cui è posizionata la propria scrivania e dalla quale ogni tanto si alza per partecipare ad una riunione, parlare con un collega o, perché no, per una pausa caffè. Durante tutti questi momenti, come è facile intuire, è possibile che il nodo entri in contatto con nodi relativi ad altri colleghi di lavoro.
\\

L'ufficio è descritto come un'unica stanza con pianta rettangolare, in cui l'unico punto di ingresso, la porta, è l'angolo in alto a sinistra e ogni persona che vi lavora ha una scrivania posizionata in un determinato punto. Non viene simulata la presenza di muri all'interno della stanza, che quindi verrà descritta come un luogo più grande del normale, in modo da simulare il tempo impiegato per superare ostacoli, nel movimento dalla scrivania ad una destinazione interna all'ufficio.
\\

Una volta entrato, l'impiegato si muove subito camminando verso la propria scrivania, dove rimane per un periodo di tempo casuale, generato utilizzando una distribuzione di Pareto. Passato questo tempo il nodo sceglie una destinazione casuale all'interno dell'ufficio, cammina fino a raggiungerla e quindi attende per un periodo di tempo casuale generato utilizzando la stessa distribuzione di Pareto prima di tornare alla propria scrivania. La ripetizione del movimento dalla scrivania ad una posizione casuale interna all'ufficio continua fino al termine della giornata lavorativa, che ha una durata impostabile in fase di configurazione.
\\

I parametri della distribuzione possono essere impostati per ogni gruppo di nodi, in modo da simulare diversi tipi di attività all'interno del luogo di lavoro, dall'insegnante che ogni ora si deve spostare in un'aula diversa, ad un commesso che non lascia mai la propria postazione per tutto l'orario lavorativo.


\subsection{Evening Activity Submodel}
Il sottomodello Evening Activity simula attività che possono essere svolte dopo lavoro, nel tardo pomeriggio - sera, come andare a fare shopping, in un bar o a mangiare in una pizzeria o ristorante. Tali attività vengono svolte in gruppo e con una probabilità configurabile, che determina se la persona torna o meno subito a casa dopo il lavoro.
\\

Al termine della giornata lavorativa, il nodo si sposta verso il proprio luogo d'incontro preferito, che è una posizione impostata all'inizio della simulazione. Una volta arrivato attende che lo raggiungano un numero di persone sufficientemente elevato per formare un gruppo e cominciare quindi l'attività. Il numero massimo e minimo di persone che possono formare un gruppo è configurabile e quando tutti i gruppi per un determinato punto di ritrovo sono al completo ne viene creato un altro.
\\

Una volta che tutti i componenti del gruppo legato all'attività sono arrivati, camminano assieme per una breve distanza verso una destinazione scelta casualmente e quindi si fermano per un tempo più lungo, generato casualmente all'interno di valori preimpostati. Una volta terminata questa pausa (finita la cena, lo shopping o la visione di un film al cinema), le varie persone si separano e tornano verso casa.

\subsection{Transport Activity Submodel}
Il Transport Activity Submodel è il sottomodello incaricato di gestire gli spostamenti dei nodi fra le diverse attività.
\\

All'inizio della simulazione ad ogni nodo viene assegnata un'auto con una probabilità configurabile. Le persone che la possiedono la utilizzeranno per tutti gli spostamenti, mentre chi ne è sprovvisto si muoverà a piedi o utilizzando un mezzo pubblico. L'eterogeneità di mezzi di trasporto utilizzati permette di simulare realisticamente i movimenti di diverse tipologie di persone ed inoltre ha impatto anche sul protocollo di routing utilizzato, in quanto nodi che si muovono utilizzando mezzi propri si sposteranno più velocemente, permettendo così il trasporto più rapido di pacchetti per lunghe distanze.
\\

A seconda del mezzo di trasporto utilizzato, quindi, il Transport Activity Submodel si rifà a tre sottomodelli distinti:
\\

\begin{description}
\item [Walking Submodel]: i nodi che non possiedono un'auto si muovono camminando lungo le strade ad una velocità costante, utilizzando l'algoritmo di Dijkstra per provare il percorso più breve dalla posizione corrente alla destinazione.
\item [Car Submodel]: i nodi che possiedono un'auto muovono più velocemente dei pedoni, durante le transizioni fra attività, ma si muovono come gli altri nodi all'interno di una singola attività. Non vengono considerati traffico e semafori durante la guida e ogni auto può portare una sola persona (il car sharing non ha ancora avuto successo nel mondo simulato).
\item [Bus Submodel]: nella città possono essere presenti più linee di trasporti pubblici (tram, autobus, funivie), ognuna delle quali viene percorsa da più mezzi ad orari predefiniti. Ogni mezzo pubblico può trasportare più persone.
\end{description}


Ogni persona che non possiede un'auto conosce una linea di mezzi pubblici e può utilizzare qualunque mezzo appartenente a quella linea. Il fatto di prendere il mezzo pubblico o di camminare dipende da un confronto di distanze euclidee: la distanza fra il luogo di partenza e quello di destinazione oppure quella fra il luogo di partenza e la fermata più vicina sommata alla distanza fra la destinazione e la fermata più vicina alla destinazione. Nel caso sia minore la prima allora il nodo camminerà fino alla destinazione, altrimenti utilizzerà i mezzi pubblici. Per fare ciò camminerà fino alla fermata più vicina, utilizzando il Walking Submodel, attenderà il primo mezzo che passerà per quella linea nella direzione corretta, utilizzerà il Bus Submodel fino alla fermata più vicina alla destinazione e quindi tornerà ad utilizzare il Walking Submodel camminando fino al punto di arrivo.
 
% ---------------------------------------------------------------------------
%: ----------------------- end of thesis sub-document ------------------------
% ---------------------------------------------------------------------------

